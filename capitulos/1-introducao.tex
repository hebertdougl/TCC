\chapter{Introdução}
\label{cap-introducao}

Nas últimas décadas houve ampliação dos recursos computacionais e do acesso à \textit{Internet}, por conseguinte houve um aumento significativo na quantidade e diversidade de conteúdos e serviços à disposição dos usuários. \citeonline{araujo2011apprecommender} menciona que um dos fatores para esse aumento é que indivíduos que anteriormente limitavam-se ao papel de consumidores de conteúdo, hoje colocam-se também na posição de produtores. Neste contexto, encontramos na Internet inúmeros exemplos de sucesso como \textit{blogs}, enciclopédias colaborativas como a Wikipedia, repositórios para compartilhamento de fotografia e vídeo, a exemplo Flickr e Youtube e redes sociais colaborativas.

\citeonline{castells2007era} afirma que tal fenômeno, comumente referenciado como \textit{Web} 2.0, ocorre porque a população acredita que pode influenciar outras pessoas, atuando no mundo por meio da sua força de vontade e utilizando seus próprios meios. Isso explica o crescimento no uso de redes sociais nos últimos anos, onde as pessoas querem produzir seus próprios conteúdos, gerando discussões construtivas sobre diversos assuntos.

% Revisar referencias, preferencialmente jogar no final da frase.

O ambiente acadêmico atual faz uso de ferramentas virtuais de aprendizagem, soluções colaborativas e redes sociais para interação. Essas ferramentas auxiliam professores e alunos em suas atividades diárias, facilitam a rápida atualização de conteúdos, facilitam o acesso a públicos geograficamente dispersos, reduz custos logísticos e administrativos e auxiliam aos alunos no desenvolvimento de capacidades de auto estudo e autoaprendizagem. Nesse contexto, deve-se buscar solução integrada, de fácil acesso e que agregue  as diversas funcionalidades necessárias a um ambiente virtual de aprendizagem.

\section{Justificativa}

A Universidade de Brasília (UnB) faz uso de ambiente virtual de aprendizagem, suportado pela ferramenta moodle, que permite a criação de cursos ``on-line'', páginas de disciplinas, grupos de trabalho e aprendizagem virtual. Essa ferramenta é um suporte tecnológico utilizada pelos professores como suporte a condução de suas disciplinas. Baseado em métodos empíricos, verifica-se que nesse contexto os alunos a utilizam, apenas para consumirem conteúdos produzidos ou responderem a estímulos oriundos dos professores.

Foi aplicado um questionário para avaliar se os alunos da Universidade de Brasília utilizam redes sociais, como o \textit{Facebook}, para o compartilhamento de recursos e informações relacionados as disciplinas cursadas na Universidade. Constatou-se que 83\% dos alunos utiliza as redes socias para tal fim, mesmo não havendo recomendação dos professores. (Resultados no Capítulo \ref{metodologia}). O resultado do questionário vaio ao encontro da afirmação de \citeonline{davis2012social}, de que os alunos de graduação querem estar nas redes sociais e que aqueles que participam de forma mais ativa nessas redes tendem a obter maior riqueza em suas relações sociais.

Para atender este novo contexto onde os usuários também querem produzir seus conteúdos e trabalharem de maneira horizontal e colaborativa, a UnB também possui uma rede colaboração livre Comunidade.UnB , desenvolvida para que alunos, professores e servidores técnico-administrativos tenham um ambiente virtual de criação e compartilhamento de conhecimento colaborativo. Essa comunidade foi implementada com base na plataforma Noosfero, que permite que o usuário tenha o seu espaço, com total flexibilidade de personalização visual e gerenciamento de conteúdo.

Neste contexto, este projeto adiciona funcionalidades para que os alunos permaneçam nas redes sociais, mas no âmbito de assuntos acadêmicos, por meio da adição de funcionalidades no Comunidade.Unb, relativas a ambientes virtuais de aprendizagem. Para propor um hibridismo entre uma rede social colaborativa da UnB e um ambiente virtual de aprendizagem, dessa maneira há necessidade de evolução da plataforma Noosfero, que não dispõe de alguns recursos necessários para esses ambientes.

\section{Objetivos}

\subsection{Objetivo Geral}

Evoluir o Noosfero para permitir que ele contenha funcionalidades de um Ambiente Virtual de Aprendizagem (AVA).

\subsection{Objetivos Específicos}

\begin{itemize}
\item Identificar as funcionalidades presentes em AVA.
\item Comparar as funcionalidades dos AVA com as disponíveis no Noosfero.
\item Selecionar as funcionalidades dos AVA a serem implementadas no Noosfero.
\item Implementar as funcionalidades selecionadas para os usuários da Comunidade.UnB.
\end{itemize}

\section{Organização do Trabalho}

Nesta seção apresenta-se como este trabalho está organizado e o que será abordado em cada um quatro dos capítulos. O trabalho se inicia com o Capítulo \ref{cap-evol-software}, no qual são abordados conceitos relacionados à manutenção e evolução de software nos métodos tradicionais e empíricos de \textit{software} além de descrições de práticas que foram utilizados no desenvolvimento deste projeto. No capítulo \ref{avas-redes-sociais} são discutidos os conceitos de redes sociais e AVA, além do levantamento de suas principais funcionalidades. No Capítulo \ref{evol-rede-social} será apresentado a arquitetura e funcionamento da plataforma Noosfero, utilizado pelo Comunidade.Unb. No Capítulo \ref{metodologia} é apresentado o estudo preliminar realizado para o desenvolvimento incluindo uma comparação de funcionalidades entre AVA e o Noosfero, afim de veirificar quais delas a plataforma carece. Além das propostas de melhoria e evolução deste trabalho para com a plataforma. No Capítulo \ref{desen-noosferAVA} são apresentados o processo realizado durante o desenvolvimento e os resultados obtidos na evolução dos \textit{plugins}. O trabalho se encerra no Capítulo \ref{conclusao} com a conclusão, assim como as limitações encontradas, além de uma discussão e trabalhos futuros, que podem incrementar esse trabalho.

\chapter{Introdução}
\label{cap-introducao}

Nas últimas décadas percebeu-se a grande difusão dos recursos computacionais e do acesso à \textit{Internet}, por conseguinte houve um aumento significativo na quantidade e diversidade de conteúdos e serviços à disposição dos usários. \citeonline{araujo2011apprecommender} menciona que um dos fatores para esse aumento é que indivíduos que anteriormente limitavam-se ao papel de consumidores de conteúdo, hoje colocam-se também na posição de produtores. Na Internet encontramos inúmeros exemplos de sucesso como \textit{blogs}, enciclopédias colaborativas como a Wikipedia, repositórios para compartilhamento de fotografia e vídeo, como Flickr e Youtube, redes sociais colaborativas, entre outros.

\cite{castells2007era} afirma que tal fenômeno, comumente referenciado como \textit{Web} 2.0, ocorre porque a população acredita que pode influenciar outras pessoas atuando no mundo por meio da sua força de vontade e utilizando seus próprios meios. Isso explica o fato do crescimento no uso de redes sociais nos últimos anos, onde as pessoas querem produzir seus próprios conteúdos gerando discussões construtivas sobre determinado assunto.

% FAlar de redes sociais

O ambiente acadêmico atual faz uso de ferramentas virtuais de aprendizagem, soluções colaborativas e redes sociais para interação. Essas ferramentas auxiliam professores e alunos em suas atividades diárias, facilitam a rápida atualização de conteúdos, facilitam o acesso a públicos geograficamente dispersos, reduz custos logísticos e administrativos e desenvolvem capacidades de auto estudo e autoaprendizagem.

Nesse contexto, deve-se buscar solução integrada, de fácil acesso, que agregue  as diversas funcionalidades necessárias a um ambiente virtual de aprendizagem.

\section{Justificativa}

A Universidade de Brasília (UnB) faz uso de ambiente virtual de aprendizagem, suportado pela ferramenta moodle, que permite a criação de cursos "on-line", páginas de disciplinas, grupos de trabalho e aprendizagem virtual. Este é um suporte tecnólogio até então utilizado pelos professores como suporte a condução de suas disciplinas. Baseado em métodos empíricos, verifica-se que nesse contexto os alunos o utiliza, apenas para consumirem conteúdos produzidos pelos professores.

Neste trabalho realizou-se um questionário afim de investigar se os alunos da Universidade de Brasília utilizam as redes sociais, como o \textit{Facebook}, para o compartilhamento de recursos e informações relacionados as disciplinas cursadas na Universidade. Verificou-se que em sua maioria os alunos utilizam as redes socias para tal fim, mesmo que os professores não recomendem o uso em suas disciplinas (Resultados no apêndice \ref{apen-quest-result}). Evidenciando a afirmação de \citeonline{davis2012social} que os alunos de graduação querem estar nas redes sociais e que aqueles que participam de forma mais ativa delas tendem a obter maior riqueza em suas relações sociais.

Para atender esse novo contexto onde os usuários também querem produzir seus conteúdos e trabalharem de maneira colaborativa, a UnB também possui uma rede colaboração livre. Esta é denominada Comunidade.UnB , desenvolvida para que alunos, professores e servidores técnico-administrativos tenham um ambiente virtual de criação e compartilhamento de conhecimento colaborativo. Essa comunidade foi implementada com base na plataforma Noosfero, que também permite que dentro de cada rede, o usuário tenha o seu espaço, com total flexibilidade de personalização visual e gerenciamento de conteúdo.

Neste contexto, este projeto visa criar instrumentos para que os alunos permaneçam nas redes sociais, mas no âmbito de assuntos acadêmicos, por meio da adição de funcionalidades no Comunidade.Unb, relativas a ambientes virtuais de aprendizagem. Para propor um hibridismo entre uma rede social colaborativa da UnB e um ambiente virtual de aprendizagem, dessa maneira há necessidade de evolução da plataforma Noosfero, que não dispõe de alguns recursos necessários para esses ambientes.

\section{Objetivos}

\subsection{Objetivo Geral}

% Manutenção e evolução de software com o objetivo de evoluir uma plataforma de rede social para suportar recursos de AVA.
% "Quais recursos de AVA e como implementá-los em uma plataforma de rede social."Manutenção e evolução de software com o objetivo de evoluir uma plataforma de rede social para suportar recursos de AVA.
Evoluir o Noosfero para permitir que ele contenha funcionalidades de um Ambiente Virtual de Aprendizagem.

\subsection{Objetivos Específicos}

\begin{itemize}
% \item Intersecções  e diferenças entre AVA e rede social
% \item Priorizar as funcionalidades de AVA q devem ser incorporadas em uma plataforma de rede social
\item Identificar as funcionalidades presentes em Ambientes Virtuais de Aprendizagem (AVA).
\item Comparar as funcionalidades dos AVA com as disponíveis no Noosfero.
\item Selecionar as funcionalidades dos AVA a serem implementadas no Noosfero.
\item Implementar as funcionalidades selecionadas.
\item Realizar testes das novas funcionalidades.
\item Disponibilizar as funcionalidades para os usuários da Comunidade.UnB.
\end{itemize}

\section{Organização do Trabalho}

Nesta seção apresenta-se como este trabalho está organizado e o que será abordado em cada um quatro dos capítulos.

O trabalho se inicia com o capítulo \ref{cap-evol-software}, no qual é abordado conceitos relacionados a manutenção e evolução de software nos métodos tradicionais e empíricos de \textit{software} além de descrições de práticas que serão utilizados no desenvolvimento deste projeto.

No capítulo \ref{avas-redes-sociais} é discutido os conceitos de ambientes virtuais de aprendizagem, bem como o levantamento de suas principais funcionalidades. E por fim uma comparação de funcionalidades entre esses ambientes e o Noosfero, afim de veirificar quais delas a plataforma carece.

No capítulo \ref{evol-rede-social} será apresentado a arquitetura e funcionamento da plataforma Noosfero, utilizado pelo Comunidade.Unb. Além das propostas de melhoria e evolução deste trabalho para com a plataforma.

E por fim no capítulo \ref{consideracoes-preliminares} serão apresentadas uma discussão sobre as funcionalidades até então implementadas, e o cronograma utilizado para a execução do projeto.
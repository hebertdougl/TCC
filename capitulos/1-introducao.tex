\chapter{Introdução}
\label{cap-introducao}
%------------------------------------------------------------------------------%
%\section{Tema}
%Manutenção e evolução em software livre.
Neste capítulo foram abordados os principais conceitos relacionados a este trabalho, mostrando o contexto e o tema que será explorado. É apresentado a justificativa e os problemas relacionados a este trabalho além dos objetivos esperados para a pesquisa. Ao final é evidenciado a organização geral do trabalho.

\section{Justificativa}


\section{Problema}


\subsection{Questão de Pesquisa}

 
\section{Objetivos}

\subsection{Objetivo Geral}

\subsection{Objetivos Específicos}


\section{Organização do Trabalho}

Este trabalho está dividido dentro de 5 outros capítulos. 

No capítulo \ref{cap-referencial-teorico} será apresentado a definição e conceitos relacionados à redes sociais, além disso os conceitos sobre software livre e a descrição sobre a ISO/IEC 14764.

No capítulo \ref{metodologia} serão apresentados as definições de como a pesquisa será realizada, além de um protocolo detalhado para realizar o levantamento dos dados para a pesquisa.

Por fim, no capítulo \ref{cap-consideracoesFinais} serão apresentadas as considerações finais para esta primeira etapa da pesquisa.
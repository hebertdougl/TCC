\chapter{Conclusão}
\label{conclusao}

Nesta primeira etapa do trabalho, foram elaboradas algumas melhorias para a evolução do Comunidade.UnB. Dentre elas, a atualização do Noosfero da Comunidade.UnB \footnote{Não possui DNS associado mas pode ser acessado pelo endereço \url{http://164.41.9.37/}}, que ainda se encontrava em \textit{Rails} 2 no Debian 6 e não dispunha de suporte pela comunidade. Da mesma forma, o tema teve que ser evoluído para ficar compatível com a nova versão da plataforma.

Como evidenciado na Seção \ref{plugin-comunidade}, realizou-se a evolução do \textit{plugin} Comunidade.UnB, permitindo que os usuários já cadastrados acessem desde que tenham cadastro no LDAP da UnB. Essa funcionalidade ainda não está habilitada mas pode ser visualizada no repositório \footnote{Repositório: \url{https://gitlab.com/unb-gama/noosfero} Branch: ldap\_unb\_plugin\_rails3 } do Noosfero para o Portal UnB Gama.

Como as demais funcionalidades a serem implementadas dizem respeito a evolução do \textit{plugin Work Assignment}, que permitirá aos professores utilizarem o Comunidade.UnB como um ambiente virtual de aprendizagem. Dessa maneira espera-se que o Comunidade.UnB transforme-se em um ambiente híbrido para os alunos e professores compartilharem o conhecimento de maneira construtiva e evolutiva.

\section{Pesquisa com alunos}
\label{pesquisa-alunos}

Para melhor prosseguirmos com este trabalho, realizamos uma pesquisa junto aos alunos da FGA sobre o uso de redes sociais em seu dia a dia e sobre o seu uso para apoio as disciplinas cursadas na universidade. Para isso, foi criado um questionário com quatro questões de múltipla escolha, e para os itens associou-se a uma escala tipo \emph{Likert} de cinco pontos: Nunca; Raramente; Às vezes; Frequentemente; Sempre (Instruções de preenchimento no Apêndice \ref{apen-inst}). Com este tipo de escala, pode-se mapear todas as respostas seguindo a lógica em que há duas alternativas negativas, duas alternativas positivas e uma intermediária.

Nesta seção, apresentamos as questões elaboradas (Apêndice \ref{apen-quest}) e as respostas coletadas através da ferramenta \textit{Google Forms}. O questionário foi aplicado na rede social \textit{Facebook}, em grupos específicos da FGA. Foram coletadas, ao todo noventa e quatro respostas voluntárias ao questionário. Abaixo segue as questões com as respectivas análises de seus resultados, que estão representados graficamente no Apêndice \ref{apen-quest-result}.

\subsection*{Com que frequência você utiliza redes sociais?}

A primeira pergunta tinha como objetivo levantar com que frequência os alunos utilizam as redes sociais. Na Figura \ref{pergunta1}, nota-se que 98\% dos alunos que responderam ao questionário utilizam as redes sociais sempre ou frequentemente, o que evidencia o fato que atualmente os estudantes dessa geração estão inseridos neste contexto.

\subsection*{Você utiliza alguma rede social como ferramenta de apoio as disciplinas?}

O intuito da segunda pergunta foi verificar se os alunos utilizam alguma ferramenta de redes social como ferramenta de apoio as disciplinas. Na Figura \ref{pergunta2}, verifica-se que apenas 16\% dos alunos às vezes utilizam e de outro ponto de vista 83\% dos alunos fazem uso da rede para tal. Demonstrando que mesmo a universidade utilizando AVA os alunos se apoiam em redes sociais para discutir e compartilhar conteúdos referentes as disciplinas cursadas.

\subsection*{Os professores incentivam o uso de redes sociais para suas disciplinas?}
% 11\% Sempre
% 19\% frequentemente
% 52\% Ás vezes
% 17\% Raramente
% 1\% Nunca

O objetivo desta pergunta foi verificar se mesmo que a universidade indique o uso de AVA se os professoes incentivam o uso das redes sociais em suas disciplinas. Na Figura \ref{pergunta3}, percebe-se que pouco mais da metade dos alunos (52 \%) tem suas respostas em um ponto intermediário do questionário, mostrando que os professoes são bem imparciais quanto a isso. Apesar que do restante, tem-se mais respostas favoráveis do que contra.

\subsection*{Mesmo que o professor não recomende o uso de redes sociais para discussão de conteúdos de suas disciplinas, você as utiliza?}
% 30\% Sempre
% 45\% frequentemente
% 20\% Ás vezes
% 4\% Raramente
% 0\% Nunca

Na quarta e última pergunta buscou-se investigar se os alunos utilizam as redes sociais como uma ferramenta de apoio as disciplinhas mesmo que os professores não recomende seu uso para tal. Dos que responderam 75\% não levam em consideração tais recomendações e fazem o uso, demonstrando que os alunos preferem estar nas redes sociais onde expõem suas opiniões a fim de promover o compartilhamento de conteúdo (Resultados na Figura \ref{pergunta4}).

Esta pesquisa exploratória mostra que mesmo com a existência de AVA os alunos a utilizam as redes sociais, que não foram criadas com esse objetivo, para a discussão de assuntos relacionados ao ambiente acadêmico.

\section{Limitações}

Ao longo deste trabalho foi realizado a evolução do \textit{plugin} Comunidade UnB que tem como objetivo prover a autenticação via LDAP. Apesar dos testes terem sido realizados no ambiente de homologação torna-se inviável colocá-lo em produção, porque a base de dados LDAP fornecida pelo CPD da FGA é a mesma utilizada para o acesso a rede \textit{wireless} e segue um mesmo padrão de senhas para todos alunos. Caso seja colocado em produção todos usuários ficariam com sérios problemas de segurança dado que não é possível alterar sua senha. Como solução deve ser solicitado acesso a base de dados única da Universidade de Brasília, que contém todos os alunos que utilizam o sistema de matrícula, visto que os testes comprovam que o \textit{plugin} está preparado para isso.

\section{Trabalhos Futuros}

Nesta seção é tratado a perspectiva de continuidade na evolução da plataforma Noosfero no contexto abordado neste trabalho, dessa maneira foi feito um levantamento das principais histórias de usuário para inspirar os próximos colaboradores.

\subsection{Evoluir calendário de atividades}

O Noosfero dispõe de um calendário para permitir a visualização dos eventos de maneira organizada. Assim sendo é proposto melhorias para que o usuário visualize todas as atividades criadas pelo professor de acordo com a data de entrega.

\subsubsection*{Histórias de usuário}
\begin{enumerate}
\item \underline{Visualizar atividades no calendário}

\textbf{Como} um usuário

\textbf{Gostaria de} visualizar todas as atividades propostas pelo professor

\textbf{Para} me inteirar sobre todas as atividades pendentes.

\subsubsection*{Cenários de uso:}
  \begin{enumerate}
  \item Integrante visualiza atividades marcadas
  \item Visualizar detalhes das atividades
  \end{enumerate}
\end{enumerate}

\begin{enumerate}
\item \underline{Notificações por email}

\textbf{Como} um usuário

\textbf{Gostaria de} receber notificações no meu email

\textbf{Para} ser alertado sobre as atividades.

\subsubsection*{Cenários de uso:}
  \begin{enumerate}
  \item Permitir que o usuário defina se deseja ser notificado via email
  \item Permitir notificações quando tempo de atividade vai expirar
  \item Notificar via email quando nova atividade for criada
  \end{enumerate}
\end{enumerate}


% Exportar CSV
% Criar média ponderada de acordo com módulos
% Fazer calculos de média
\chapter{Considerações preliminares}
\label{consideracoes-preliminares}

Nesta primeira etapa do trabalho foram elaboradas algumas melhorias fundamentais para a evolução do Comunidade.UnB. Dentre elas destaca-se a atualização da plataforma Noosfero, que ainda se encontrava em \textit{Rails} 2 e não dispunha de suporte pela comunidade, em um novo ambiente de homologação disponibilizado pelo CPD. Da mesma forma o tema teve que ser evoluído porque não dava mais suporte a nova versão da plataforma.

Como evidenciado na seção \ref{plugin-comunidade} realizou-se a evolução do \textit{plugin} Comunidade.UnB, permitindo que os usuários já cadastrados acessem desde que tenham cadastro no LDAP da UnB. Esta funcionalidade ainda não está habilitada mas pode ser visualizada no repositório \footnote{Repositório: \url{https://gitlab.com/unb-gama/noosfero} Branch: ldap\_unb\_plugin\_rails3 } do Noosfero para o Portal UnB Gama.

Após o desenvolvimento desta funcionalidade surgiu a discussão sobre a possibilidade de autenticação dos usuários no Comunidade.UnB com os mesmos dados utilizados em redes sociais como o \textit{Facebook} ou \textit{Google+}. Esta é uma idéia bastante pertinente já que grande parte dos alunos utilizam estas redes para construção de relações sociais, além disso as utilizam como apoio as disciplinas da universidade. Como pode ser visto no apêndice \ref{apen-quest-result}, que contém o resultado do questionário aplicado aos alunos.

Partindo desse pressuposto pretende-se investigar a viabilidade do uso desse modelo de autenticação onde acredita-se que irá atrair mais usuários ao Comunidade.UnB, devido a facilidade de cadastro. Essea funcionalidade já está disponível em alguns \textit{websites} que utilizam a plataforma Noosfero, facilitando sua possível implantação.

Com as demais funcionalidades a serem implementadas dizem respeito a evolução do \textit{plugin Work Assignment}, que permitirá aos professores utilizarem o Comunidade.UnB como um ambiente virtual de aprendizagem. Dessa maneira espera-se que o Comunidade.UnB transforme-se em um ambiente híbrido para os alunos e professores compartilharem o conhecimento de maneira construtiva e evolutiva.

\section{Cronograma}

O cronograma para o trabalho realizado neste TCC é apresentado na Tabela \ref{cronograma}. As atividades planejadas são:

% Cronograma
\begin{enumerate}
\item Verificar uso autenticação por redes sociais
\item Disponibilizar novo método de autenticação aos usuários
\item Desenvolver histórias
\begin{enumerate}
\item Definir tempo restante
    % Definir tempo
    % Modificar tempo
    % Permitir a entrega de atividades após o período
\item Tempo restante de atividades
    % visualizar tempo
\item Professor gerencia notas
    % Definir grupo de atividades
    % Visualizar notas de todos os alunos de uma determinada atividade
    % Visualizar notas de todos ao alunos de um grupo de atividades
\item Atribuir notas aos alunos
    % Atribuir notas
    % Alterar notas
\item Publicar notas aos alunos
    % Disponibilizar notas de uma determinada atividade
    % Omitir notas de uma determinada atividade
\item Aluno visualiza notas
    % Visualizar cursos com notas disponíveis
    % Detalhar notas de cada curso
\end{enumerate}
\item Disponibilização de funcionalidades no Comunidade.UnB.
\item Escrita do trabalho de conclusão de curso 2
\end{enumerate}

\begin{table}[h]
    \centering

    \begin{tabular}{|c|c|c|c|c|c|c|}
        \hline
        \textbf{Atividades} & \textbf{Jul 2015} & \textbf{Ago 2015} & \textbf{Set 2015}
        & \textbf{Out 2015} & \textbf{Nov 2015} & \textbf{Dez 2015} \\
        \hline\hline

        \hline
        1   & • &   &   &   &   &   \\

        \hline
        2   & • &   &   &   &   &   \\

        \hline
        3. a)   & • &   &   &   &   &   \\

        \hline
        3. b)   &   & • & • &   &   &   \\

        \hline
        3. c)   &   & • & • &   &   &   \\

        \hline
        3. d)   &   &   & • & • &   &   \\

        \hline
        3. e)   &   &   & • & • &   &   \\

        \hline
        3. f)   &   &   &   & • & • &   \\

        \hline
        4   &   &   &   & • & • &   \\

        \hline
        5   &   &   &   &   &   & • \\

        \hline
    \end{tabular}

    \caption{Cronograma de atividades para TCC2}
    \label{cronograma}
\end{table}
\chapter{Conclusão}
\label{conclusao}

No decorrer deste trabalho de conclusão de curso, foram realizadas atividades que englobam as principais áreas de conhecimento da Engenharia de Software, desde o levantamento de requisitos à implementação e implantação das funcionalidades desenvolvidas. Houve contribuições no contexto da manutenção e evolução de uma plataforma real de software livre, onde foi priorizado a evolução do conhecimento de forma colaborativa com os membros do laboratório avançado de pesquisa e produção de software.

O desenvolvimento proporcionou a participação em um processo distribuído e colaborativo de desenvolvimento de software, dado que o desenvolvimento foi realizado em conjunto com uma comunidade de software livre com equipes espalhadas pelo país.

No trabalho foram identificadas e detalhadas as principais funcionalidades de um ambiente virtual de aprendizagem comparando-as com as disponíveis no Noosfero. Dessa maneira foram selecionadas as mais importantes para os usuários, implementadas e integradas a rede Comunidade.UnB. 

A rede colaborativa Comunidade.UnB foi atualizada para uma versão mais recente que provém de novas funcionalidades e melhores níveis de segurança, além de habilitar o protocolo de segurança SSL para manter a integridade das informações dos usuários. Com evolução dos \textit{plugins} a autenticação, em ambiente de testes, é realizada de acordo com a base de dados LDAP e agrega ao professor funcionalidades que proporcionam melhor controle das atividades disponibilizadas e a atribuição de notas.

Portanto a Rede Comunidade.UnB está atualizada com a última versão de pacotes do Noosfero, 
Dessa maneira espera-se que o Comunidade.UnB transforme-se em um ambiente híbrido para os alunos e professores compartilharem o conhecimento de maneira construtiva e evolutiva.

\section{Limitações}

Ao longo deste trabalho foi realizado a evolução do \textit{plugin} Comunidade UnB que tem como objetivo prover a autenticação via LDAP. Apesar dos testes terem sido realizados no ambiente de homologação torna-se inviável colocá-lo em produção, porque a base de dados LDAP fornecida pelo CPD da FGA é a mesma utilizada para o acesso a rede \textit{wireless} e segue um mesmo padrão de senhas para todos alunos. Caso seja colocado em produção, todos usuários ficariam com problemas de segurança dado que não é possível alterar a senha na base LDAP. Como solução deve ser solicitado acesso a base de dados única da Universidade de Brasília, que contém todos os alunos que utilizam o sistema de matrícula, visto que os testes comprovam que o \textit{plugin} está preparado para isso.

\section{Trabalhos Futuros}

Nesta seção é tratado a perspectiva de continuidade na evolução da plataforma Noosfero no contexto abordado neste trabalho, dessa maneira foi feito um levantamento das principais histórias de usuário para inspirar os próximos colaboradores.

\subsection*{Evoluir calendário de atividades}

O Noosfero dispõe de um calendário para permitir a visualização dos eventos de maneira organizada. Assim sendo é proposto melhorias para que o usuário visualize todas as atividades criadas pelo professor de acordo com a data de entrega.

\subsubsection*{Histórias de usuário}
\begin{enumerate}
\item \underline{Visualizar atividades no calendário}

\textbf{Como} um usuário

\textbf{Gostaria de} visualizar todas as atividades propostas pelo professor

\textbf{Para} me inteirar sobre todas as atividades pendentes.

\subsubsection*{Cenários de uso:}
  \begin{enumerate}
  \item Integrante visualiza atividades marcadas
  \item Visualizar detalhes das atividades
  \end{enumerate}
\end{enumerate}

\subsection*{Mecanismos de notificações}

\begin{enumerate}
\item \underline{Notificações por email}

\textbf{Como} um usuário

\textbf{Gostaria de} receber notificações no meu email

\textbf{Para} ser alertado sobre as atividades.

\subsubsection*{Cenários de uso:}
  \begin{enumerate}
  \item Permitir que o usuário defina se deseja ser notificado via email
  \item Permitir notificações quando tempo de atividade vai expirar
  \item Notificar via email quando nova atividade for criada
  \end{enumerate}
\end{enumerate}

\subsection*{Ferramentas para notas}

\begin{enumerate}
\item \underline{Ponderar notas}

\textbf{Como} um professor

\textbf{Gostaria de} ponderar as notas dos módulos da disciplina

\textbf{Para} definir a menção dos alunos.

\subsubsection*{Cenários de uso:}
  \begin{enumerate}
  \item Definir quais os módulos fazem parte da média do aluno
  \item Selecionar quais as atividades fazem parte da avaliação
  \item Criar média ponderada para as atividades
  \item Exportar médias em formato CSV
  \end{enumerate}
\end{enumerate}

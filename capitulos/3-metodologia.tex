\chapter{Metodologia}
\label{metodologia}
%-Quais são os sujeitos da pesquisa?
%-Instrumentos e equipamentos que serão utilizados para os procedimentos?
%-Quais são os procedimentos de coleta de dados?

Uma pesquisa científica pode ser entendida como a realização concreta de uma investigação planejada e desenvolvida de acordo com as normas consagradas pela metodologia científica. Metodologia científica entendida como um conjunto de etapas ordenadamente dispostas que você deve vencer nas investigação de um fenômeno, é utilizada para determinação das formas que serão utilizadas para reunir os dados necessários para o desenvolvimento de um trabalho científico. Inclui a escolha do tema, o planejamento da investigação, o desenvolvimento metodológico, a coleta e a tabulação de dados, a análise dos resultados, a elaboração das conclusões e a divulgação dos resultados \cite{moresi2003metodologia}.

\section{Classificação da pesquisa}
\label{class-da-pesquisa}
A classificação dessa pesquisa foi realizada de acordo com o trabalho realizado \citeonline{moresi2003metodologia} como pode ser visualizada abaixo:

Quanto a \textbf{natureza} é uma pesquisa básica pois visa apenas um estudo aprofundado pelo tema sem uma aplicação prática prevista.

Quanto a \textbf{forma de abordagem do problema} é uma pesquisa quantitativa pois o método de pesquisa utilizado busca atingir os resultados, baseado em uma análise de várias comunidades de software livre.

Quanto aos \textbf{fins} a pesquisa será exploratória já que é uma área onde há pouco conhecimento sistematizado. A pesquisa exploratória tem como objetivo proporcionar maior familiaridade com o problema, com vistas a torná-lo mais explícito ou a construir hipóteses. A grande maioria dessas pesquisas envolve: (a)levantamento bibliográfico; (b) entrevistas com pessoas que tiveram experiências práticas com o problema pesquisado; e (c) análise de exemplos que estimulem a compreensão \cite{gil2002}.

Quanto aos \textbf{meios} é uma pesquisa bibliográfica, pois compreendeu o estudo sistematizado a partir de conhecimentos disponibilizados em artigos, livros, periódicos, jornais, acervos de dados eletrônicos, entre outros. E é também um \emph{survey} para obtenção dos dados ou informações sobre caracterísitcas das comunidades de software livre.

\section{Fases e atividades da pesquisa}
\label{atv-da-pesquisa}
A pesquisas está dividida em cinco fases, e para cada uma dessas foram definidas as atividades que devem ser desenvolvidas para atingir o objetivo de cada uma delas. A divisão pode ser visualizada na tabela \ref{tab:fas-atv-pesq}.

\begin{table}[H]
\center
\begin{tabular}{|l|l|}
\hline
\textbf{Fases} & \textbf{Atividades} \\ \hline
\multirow{4}{*}{Planejamento} 
		& 1- Definir tema \\
		& 2- Definir problema \\
		& 3- Definir justificativa \\
		& 4- Definir objetivos \\ \hline
\multirow{2}{*}{Revisão teórica}
		& 1- Definir os assuntos que deverão ser aborados \\
		& 2- Realizar referencial teórico\\ \hline
\multirow{2}{*}{Análise metodológica}
		& 1- Definir qual método será utilizado \\
		& 2- Elaborar protocolo para o método \\ \hline
\multirow{6}{*}{Execução} 
		& 1- Refinar questionário \\
		& 2- Colocar questionário na ferramenta\\
		& 3- Selecionar comunidaes de software livre \\
		& 4- Aplicar questionários \\
		& 5- Coletar os dados na ferramenta \\
		& 6- Analisar os dados obtidos \\ \hline
\multirow{2}{*}{Finalização} 
		& 1- Documentar todos os resultados obtidos \\
		& 2- Elaborar conclusão \\ \hline
\end{tabular}
\caption{Fases e atividades da pesquisa}
\label{tab:fas-atv-pesq}
\end{table}



\subsection{Descrição das atividades}
\label{descr-atv}
Realizou-se a descrição de todas as atividades envolvidas no plano de pesquisa, as atividades da fase de \textbf{Planejamento} podem ser visualizadas na tabela \ref{tab:fas-atv-plan}.
\begin{table}[H]
\center
\begin{tabular}{|l|p{8cm}|}
\hline
\textbf{Atividade} & \textbf{Descrição} \\ \hline
Definir tema & O tema para a pesquisa é definido de acordo com as preferências do pesquisador.\\ \hline
Definir problema & É escolhido um problema relacionado ao tema abordado.\\ \hline
Definir justificativa & São especificados as justificativas pela escolha do tema proposto. \\ \hline
Definir objetivos & São definido objetivos para que o problema e as questões de pesquisa possam ser alcançadas.\\ \hline
\end{tabular}
\caption{Descrição atividades da fase de planejamento}
\label{tab:fas-atv-plan}
\end{table}

As atividades da fase de \textbf{Revisão teórica} podem ser visualizadas na tabela \ref{tab:fas-atv-rev}.

\begin{table}[H]
\center
\begin{tabular}{|l|p{8cm}|}
\hline
\textbf{Atividade} & \textbf{Descrição} \\ \hline
Definir os assuntos que deverão ser abordados & Definir quais são os assuntos que devem ser levantadas afim de se atingir os objetivos do trabalho.\\ \hline
Realizar referencial teórico & Elaborar e documentar todo o embasamento teórico necessário para a execução da pesquisa. \\ \hline
\end{tabular}
\caption{Descrição atividades da fase de revisão teórica}
\label{tab:fas-atv-rev}
\end{table}

As atividades da fase de \textbf{Análise metodológica} podem ser visualizadas na tabela \ref{tab:fas-atv-plan}.

\begin{table}[H]
\center
\begin{tabular}{|l|p{8cm}|}
\hline
\textbf{Atividade} & \textbf{Descrição} \\ \hline
		Definir qual método será utilizado & Definir qual o método de pesquisa que será utilizado para atingir os objetivos propostos. \\ \hline
		Elaborar protocolo para o método & Desenvolver um protocolo que será utilizado para guiar e auxiliar na condução da pesquisa. \\ \hline
\end{tabular}
\caption{Descrição atividades da fase de análise metodológica.}
\label{tab:fas-atv-ana-met}
\end{table}

As atividades da fase de \textbf{Execução} podem ser visualizadas na tabela \ref{tab:fas-atv-exec}.

\begin{table}[H]
\center
\begin{tabular}{|l|p{8cm}|}
\hline
\textbf{Atividade} & \textbf{Descrição} \\ \hline
		Refinar questionário & O questionário será submetido a uma análise por especialistas e modificado de acordo com as observações.\\ \hline
		Colocar questionário na ferramenta & Incluir todas as questões levantadas em uma ferramenta de auxilio.\\ \hline
		Selecionar comunidades de software livre & Realizar uma busca de todas as comunidades de software livre que atendem aos critérios levantados. \\ \hline
		Aplicar questionários & Disponibilizar o questionário com suas instruções de uso para todas as comunidades selecionadas \\ \hline
		Coletar os dados na ferramenta & Realizar a coleta de todos os dados levantadas pela ferramento em uso. \\ \hline
		Analisar os dados obtidos & Realizar a análise de todos os dados coletados. \\ \hline
\end{tabular}
\caption{Descrição atividades da fase de execução}
\label{tab:fas-atv-exec}
\end{table}

As atividades da fase de \textbf{Finalização} podem ser visualizadas na tabela \ref{tab:fas-atv-fin}.

\begin{table}[H]
\center
\begin{tabular}{|l|p{8cm}|}
\hline
\textbf{Atividade} & \textbf{Descrição} \\ \hline
		Documentar todos os resultados obtidos & Evidenciar e documentar todos os resultados obtidos durante a coleta de dados. \\ \hline
		Elaborar conclusão & Após a análise realizar a conclusão da pesquisa realizada. \\ \hline
\end{tabular}
\caption{Descrição atividades da fase de finalização}
\label{tab:fas-atv-fin}
\end{table}

\section{Estudo de Caso}

\section{Cronograma}
\label{cronograma}

O cronograma apresentado na tabela \ref{tab:cronograma} contempla todas as atividades restantes para a conclusão da pesquisa.

\begin{table}[H]
\center
\begin{tabular}{|l|p{8cm}|}
\hline
\textbf{Mês} & \textbf{Atividades} \\ \hline
\multirow{2}{*}{Dezembro} 
		& Refinar questionário \\ 
		& Colocar questionário na ferramenta\\ \hline
\multirow{2}{*}{Janeiro}
		& Selecionar comunidades de software livre \\
		& Aplicar questionários \\ \hline
\multirow{1}{*}{Fevereiro}
		& Aplicar questionários \\ \hline
\multirow{2}{*}{Março}
		& Coletar os dados na ferramenta \\
		& Analisar os dados obtidos \\ \hline
\multirow{2}{*}{Abril} 
		& Analisar os dados obtidos \\
		& Documentar todos os resultados obtidos \\ \hline
\multirow{1}{*}{Maio}    		
		&Elaborar conclusão \\ \hline
\end{tabular}
\caption{Cronograma para execução da pesquisa}
\label{tab:cronograma}
\end{table}
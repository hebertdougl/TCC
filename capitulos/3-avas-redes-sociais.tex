\chapter{Ambiente Virtual de Aprendizagem e Plataformas de Redes Sociais}
\label{avas-redes-sociais}

Neste capítulo serão apresentadas os conceitos relacionados aos ambientes virtuais de aprendizagem e redes sociais. Além disso há uma comparação dos principais AVA levantados nessa pesquisa com a plataforma Noosfero.

\section{Ambiente virtual de aprendizagem}
\label{ava}

No atual cenário, em que vivemos a tecnologia da informação está cada vez mais presente em nossa rotina, seja com o uso de celulares, tablets, televisores entre outros e utilizá-la no ambiente educacional pode ser uma grande vantagem de integração à sociedade do conhecimento. Nesse cenário ambientes virtuais de aprendizagem tem sido cada vez mais usados nas mais diversas instituições (acadêmicas, empresariais e tecnológicas) com o objetivo de apoiar o ensino à distância.

Os ambientes virtuais de aprendizagem \footnote{Em inglês \textit{Virtual Learning Environment} (VLE)} são sistemas que tem a funcionalidade de \textit{software} para a comunicação entre computador e métodos de entrega de material de cursos online. Muitos desses ambientes permitem simular a sala de aula física em um meio virtual contudo existem ambientes que além de realizar este papel fazem uso da tecnologia para propiciar aos envolvidos novas ferramentas que potencializem a comunicação e facilitem a aprendizagem \cite{schlemmer2005ambiente}.

% Esses últimos procuram suportar uma grande e variada gama de estilos de aprendizagem e objetivos, encorajando a colaboração, a aprendizagem baseada em pesquisa, além de promover compartilhamento e reuso dos recursos.

Os AVA são utilizadas por várias instituições de ensino do mundo. O mercado apresenta uma grande variedade de \textit{softwares} proprietários e livres. Entretanto, apesar dessa diversidade, \citeonline{behar2009avaliaccao} destaca que muitas características e ferramentas são comuns entre eles:
\begin{itemize}
\item permitem acesso restrito a usuários previamente cadastrados;
\item disponibilizam espaço para a publicação de material do professor;
\item disponibilizam espaço destinado ao envio/armazenamento de tarefas realizadas pelos alunos;
\item disponibilizam um conjunto de ferramentas de comunicação síncrona e assíncrona\footnote{Como \textit{chat} e fórum de discussões};
\item disponibilizam correio eletrônico (\textit{e-mail}) e mural de recados.
\end{itemize}

% Linkar parágrafos
Além dessas características, \citeonline{aguado2013dimensoes} ressalta que os principais componentes encontrados consistem de:
\begin{itemize}
\item disponibilização de conteúdo de ensino;
\item mapeamento do conteúdo do curso (quebra do conteúdo em sessões que podem ser acessadas e cumpridas);
\item disponibilização de atividades de interação e avaliação;
\item acompanhamento do desempenho do aluno;
\item suporte online;
\item links de internet para conteúdos externos;
\item identificações e permissões de acesso.
\end{itemize}

Apesar dos autores supracitados relatarem as principais características e componentes dos AVA, constata-se que nem todos oferecem as mesmas funcionalidades como pode ser verificado na seção \ref{comparacao-ava} desse trabalho. O escopo das opções disponíveis nos sistemas nem sempre é o mesmo, evidenciando a razão do alto número de sistemas no mercado, pois os produtos possuem pontos fortes e fracos, e por isso, nem sempre as funcionalidades atendem aos diversos tipos de instituições e seus respectivos objetivos \cite{aguado2013dimensoes}.

Segundo \citeonline{chin2003virtual}, as funcionalidades oferecidas pelo AVA facilitam e fortalecem o aprendizado de conceitos dados presencialmente. Mesmo que os AVA apresentem características principais do ensino tradicional em sala de aula, \citeonline{chin2003virtual} propõe que os AVA não devem ser vistos como substitutos à aula presencial, uma vez que sempre deverá existir a interação humana física no processo educacional.

Dentre os vários sistemas disponíveis no mercado alguns necessitam de instalação no computador cliente, porém, a maioria é acessível pela internet, onde a instalação é realizada apenas no servidor. Das ferramentas encontradas atualmente destacam-se as tradicionalmente utilizadas como o Moodle ~\footnote{\textit{Modular Object-Oriented Dynamic Learning Environment}. Disponível em: \url{https://moodle.org/}} e o BlackBoard~\footnote{Disponível em: \url{http://www.blackboard.com/}} detalhadas na seção \ref{comparacao-ava}.

%
%A expressão Ambiente Virtual de Aprendizagem, de acordo com \cite{almeidaEAD}, “relaciona-se à sistemas computacionais, destinados ao suporte de atividades mediadas pelas tecnologias de informação e comunicação”. Permitem integrar múltiplas mídias e recursos, apresentam informações de maneira organizada, proporcionam interações entre pessoas e objetos de conhecimento, visando atingir determinados objetivos.
%
% Dentro deste contexto, ao se tratar a qualidade dessas ferramentas é necessário considerar que o item mais importante a ser avaliado é o critério didático-pedagógico, afinal, este é um produto da área da educação que deve auxiliar os alunos e envolvidos a aquisição de algum conhecimento, porém não se pode deixar de lado outros aspectos importantes de usabilidade técnica que podem inviabilizar o uso da ferramenta ou mesmo, torna-lo difícil a ponto da aprendizagem deixar de ser prazerosa e assim a tecnologia não estará cumprindo seu principal papel que é ser um grande facilitador do processo de aprendizagem.
%
\section{Redes sociais}
\label{rede-social}

Para entender o conceito de redes sociais é importante destacar que são apenas uma camada das mídias sociais. É comum vermos a utilização deste termo para todos os tipos de mídias sociais mediadas por um computador. As tecnologias de mídias sociais \footnote{Termo em inglês, \textit{Social Media Technology} (SMT)} podem ser definidas de forma abrangente como a totalidade de produtos e serviços digitais disponibilizados online, como aplicações \textit{web} e \textit{mobile} que permitem aos indivíduos criar e compartilhar conteúdos onde sua principal fonte são os usuários, através da comunicação em várias vias \cite{davis2012social}.

As definições de rede utilizadas neste trabalho estão relacionadas com três propostas. A primeira é de acordo com \citeonline{emirbayer1994network} que definem rede como um conjunto de relações ou ligações entre um conjunto de atores, no caso atores são os elos às pessoas que se comunicam em uma dada rede. No mesmo sentido, \citeonline[p. 73]{costa2004redes} a define com uma forma de organização caracterizada fundamentalmente pela sua horizontalidade, isto é, pelo modo de inter-relacionar os elementos sem hierarquia. E por fim, mas não menos importante, faz-se uso da definição proposta por \citeonline{tomael2005redes} no qual rede é uma estrutura não-linear, descentralizada, flexível, dinâmica, sem limites definidos e autoorganizáveis. Destacando que uma rede estabelece-se por relações horizontais de cooperação.

Segundo \cite{marteleto2001analise}, as redes sociais derivam dos conceitos de rede e representam um conjunto de participantes autônomos, que buscam unir suas idéias e recursos em torno de valores e interesses compartilhados. A autora ressalta a ideia de compartilhamento de valores e interesses que, para promover o fortalecimento da rede, dependem do compartilhamento da informação e do conhecimento.

Nesse contexto, os autores \citeonline{ellison2007social} definem sites de redes sociais como serviços \textit{web} que permitem a seus usuários criarem perfis e, por meio deles, conectarem com outros usuários, propiciando a busca e o cruzamento de informações dentro da sua lista de conexões.

Fundamentado nestes principais conceitos, este trabalho está focado em sites de redes sociais, que possuem o objetivo de promover a interação em torno das colaborações, isto é, apoiada no entendimento de redes de colaboração. Isto se aplica ao contexto de uma rede social de nicho de uma universidade, como explicitado na seção \ref{comunidade-unb} onde é descrita a proposta de disponibilização de uma rede social para a Universidade de Brasília (UnB). A rede da UnB permite que as pessoas entrem na rede para acompanharem uma disciplina, um projeto ou um determinado grupo de trabalho. Os alunos também podem explorar a rede para encontrar comunidades e conteúdos de seus interesses.

\citeonline{bucher2013rede} menciona que esta prática adotada por redes de colaboração não é possível em redes monopolistas e centralizadoras, porque em geral seus conteúdos estão dipostos de forma fragmentada ou são controlados. Isto demonstra que o uso de uma rede de colaboração dentro de universidades é uma escolha mais plausível para a difusão do conhecimento.

Apoiado nas vantagens proporcionadas pelas redes sociais de colaboração e sua implantação na UnB, busca-se neste trabalho associá-las aos conceitos dos ambientes virtuais de aprendizagem tratados na seção \ref{ava}. A proposta visa a criação de funcionalidades que viabilizem o uso de uma plataforma de redes sociais como um ambiente de apoio aos ambientes virtuais de aprendizagem.

\section{Comparação entre Noosfero e AVA}
\label{comparacao-ava}

Como o objetivo do trabalho é a evolução de plataforma de redes sociais para um ambiente virtual de aprendizagem, a pesquisa realizada fundamenta-se no levantamento das principais funcionalidades que os AVA atuais utilizam, para verificar quais a plataforma o Noosfero ainda não possui. Desse modo surgiu a necessidade do levantamento das principais ferramentas AVA utilizadas atualmente pelas universidades e instituições de ensino, para proporcionar o mapeamento destas funcionalidades.
%
Os AVA selecionados para comparação são \textbf{BlackBoard}, \textbf{Sakai}, \textit{Modular Object-Oriented Dynamic Learning Environment} (\textbf{Moodle \footnote{Disponível em: \textit{ \url{https://moodle.org/}}}}) e \textit{TelEduc}.

O BlackBoard~\footnote{Disponível em: \url{http://www.blackboard.com/}}, é um dos maiores ambientes virtuais de aprendizagem utilizados atualmente por um, total de 72\% das TOP 200 melhores universidades do mundo \cite{blackboard}. Para utilização desta plataforma é necessário a aquisição de licenças por parte da instituição que pretende implementá-la. A plataforma tem total suporte da empresa que a mantém e não permite a modificação de sua estrutura interna pelos seus utilzadores. Os ajustes e melhorias de funcionalidades são realizados através do mapeamento dos comentários e sugestões dos usuários.

O Moodle, é um software livre criado em 2001 pelo educador cientista computacional Martin Dougiamas. É desenvolvido na linguagem de programação PHP e desta maneira pode ser instalado em qualquer sistema operacional que tenha suporte à linguagem. Para o seu uso podem ser utilizadas várias bases de dados com suporte a ODBC\footnote{Padrão para acesso a sistemas gerenciadores de bancos de dados}, tais como MySQL, PostgreSQL e Oracle.

O projeto Sakai\footnote{Disponível em: https://sakaiproject.org/ } ou a comunidade Sakai desenvolve e distribui o software livre Sakai como ambiente virtual de aprendizagem para colaboração e ensino para educadores, por educadores. A plataforma foi desenvolvido em linguagem Java e pode ser executado em várias plataformas diferentes como Linux, Unix, Windows e MAC. Suporta banco de dados MySQL e Oracle. As instituições que utilizam esta plataforma são abrangentes e estão listadas no site da comunidade \footnote{Disponível em: https://www.sakaiproject.org/community}.

O \textbf{TelEduc}\footnote{Disponível em: \url{http://www.teleduc.org.br/}} é um ambiente virtual de aprendizagem, cujo desenvolvimento iniciou em 1997, a partir de uma dissertação de mestrado de Alessandra de Dutra e Cerceau do Instituto de Computação da Universidade Estadual de Campinas\footnote{IC/UNICAMP. Disponível em: \url{http://www.ic.unicamp.br/}}. Foi objeto de pesquisa e desenvolvimento de um projeto coordenado pela Profa Dra Heloísa Vieira da Rocha\footnote{\url{http://lattes.cnpq.br/6985892121344767}} até 2012. Desde então, o projeto vem sendo evoluído para implementar ajustes e novas funcionalidades segundo \cite{rocha2002ambiente}, visando implementar necessidades relatadas por seus Usuários. O projeto é mantido pelo Núcleo de Informática Aplicada à Educação \footnote{NIED/UNICAMP. Disponível em:\url{http://www.nied.unicamp.br/}} da UNICAMP.

O TelEduc é um software livre desenvolvido na linguagem de programação PHP \footnote{Disponível em: \url{http://php.net/}} e JavaScript com banco de dados o MySQL \footnote{Disponível em: \url{https://www.mysql.com/}} para ambientes Linux. Em seu lançamento tornou-se um dos softwares mais utilizados para apoiar a educação à distância nas mais diversas áreas. Seus principais usuários são as universidades públicas e privadas, que utilizam a ferramenta para atividades educacionais, disponibilizando materiais, dando suporte a comunicação e interação entre os participantes.

Foram criadas tabelas comparativas (Tabelas \ref{tab:conteudo-atribuicao}, \ref{tab:curso}, \ref{tab:ferramentas}, \ref{tab:teste}, \ref{tab:permissoes-principal}) entre as ferramentas selecionadas e a plataforma Noosfero, desse modo fica mais evidente identificar quais funcionalidades a plataforma Noosfero carece. E vale ressaltar que não é objetivo deste trabalho esclarecer ou afirmar qual é a melhor ferramenta, mas verificar as principais funcionalidades dos AVA que são utilizados e aceitos pelas universidades.

As funcionalidades incluídas nas tabelas foram obtidas da literatura apresentada na seção \ref{ava}, e em uso aos ambientes onlines de demonstração, contando ainda, com o auxílio de mecanismos de ajuda fornecidos pelas ferramentas. Assim sendo, as funcionalidades selecionadas foram divididas em oito categorias:
\begin{itemize}
\item \textbf{Conteúdo:} relacionadas à criação e manutenção de conteúdos publicados pelos usuários.
\item \textbf{Atribuição:} permitem que os usários carreguem arquivos do seu computador local para a plataforma.
\item \textbf{Ferramentas:} mecanismos de apoio ao gerenciamento do conteúdo e ambiente;
\item \textbf{Teste/Quiz:} permitem a realização de avaliações e acompanhamento de resultados dos alunos;
\item \textbf{Comunicação:} possibilitam a comunicação síncrona e assíncrona entre os participantes de um curso;
\item \textbf{Curso:} fundamentais para organização dos cursos;
\item \textbf{Permissão e papéis:} utilizadas para gerenciar o ambiente, definindo papéis e autorizações de acesso ao ambiente;
\item \textbf{Página principal:} são as principais funcionalidades evidenciadas na página inicial de cada usuário.
\end{itemize}

Das categorias listadas na Tabela \ref{tab:conteudo-atribuicao}, podemos destacar que o Noosfero carece do SCORM \footnote{Um modelo de referência seja, conjunto unificado de especificações para a disponibilização de conteúdos e serviços de e-learning \cite{de2006objetos}.} e IMS-Content-Package \footnote{Padrão que permite exportar o conteúdo de um sistema de gerenciamento de conteúdo de aprendizagem ou repositório digital.} que são funcionalidades importantes para os AVA, pois permitem a importação de conteúdos de outras ferramentas proporcionando flexibilidade para o usuário.

Verifica-se que o Noosfero possui funcionalidades básicas de comunicação, mas não possibilita a comunicação síncrona (Bate-papo). Possui ainda, fórums que normalmente são utilizados para a divulgação de notícias e dúvidas, entretanto, não permite a assinatura de tópicos para o recebimento de notificações.

Como o Noosfero não é um ambiente virtual de aprendizagem, é possível verificar na Tabela \ref{tab:teste} que ele não possibilita a criação de questões para obter a resposta de usuários. Também não conta com a funcionalidade que permita ao professor atribuir as notas e acompanhar a situação de cada aluno. O Noosfero possui um \textit{plugin} denominado \textit{Work Assignment} que permite aos usuários enviarem arquivos para o servidor mas não possibilita que o moderador estabeleça uma pontuação para os mesmos.

\begin{landscape}

\begin{table}[H]
	\centering
	\begin{tiny}
	\begin{tabular}{C{4cm}C{3.2cm}C{3.2cm}C{3.2cm}C{3.2cm}C{3.2cm}}
	\toprule
	\textbf{Funcionalidades\textbackslash Sistema} & \textbf{BlackBoard 9.1} & \textbf{Moodle} & \textbf{TelEduc} & \textbf{Sakai} & \textbf{Noosfero}\\
	\midrule
	\multicolumn{6}{c}{\textbf{Conteúdo}}\\
	\midrule
	Diretórios/Pastas & Sim & Sim & Sim & Sim & Sim \\
	Criar/Extrair arquivos de pastas & Sim & Sim & Sim & Sim & Sim \\
	Editor HTML & Sim & Sim & Sim & Sim & Sim \\
	Filtro de vários idiomas & Não & Sim & \textbf{Não} & \textbf{Não} & Sim \\
	Audio & Sim {\tiny (incorporado no plugin do QuickTime)} & Sim & \textbf{Não} & Sim &  \\
	Vídeo & Sim & Sim & \textbf{Não} & Sim & Sim \\
	SCORM & Sim & Sim & \textbf{Não} & Sim & \textbf{Não} \\
	IMS-Content-Package & Sim & Sim &  &  & \textbf{Não} \\
	\midrule
	\multicolumn{6}{c}{\textbf{Atribuição}}\\
	\midrule
	Upload de um arquivo & Sim & Sim & Sim & Sim & Sim \\
	Upload de vários arquivos & Sim & Sim & \textbf{Não} & Sim & Sim \\
	Texto online & Sim(quiz) & Sim & Sim & Sim & Sim \\
	\midrule
	\multicolumn{6}{c}{\textbf{Comunicação}}\\
	\midrule
	Enviar e-mail & Sim & Sim & Sim& Sim   & Sim    \\
	Enviar mensagens& Sim & Sim & Sim& Sim   & Sim {\tiny(mural)}\\
	Bate papo& Sim & Sim & Sim& Sim   & \textbf{Não}    \\
	Fórum de discussão& Sim & Sim & Sim& Sim   & Sim    \\
	Alterar tipo de fórum & Sim & Sim & Sim& Sim   & Sim \\
	Fórum:   & Sim & Sim & Sim& Sim   & Sim    \\
	-postagem anônima & Sim & \textbf{Não} & \textbf{Não}& \textbf{Não}   & \textbf{Não}    \\
	-anexar um arquivo& Sim & Sim & \textbf{Não}& Sim   & \textbf{Não}    \\
	-os participantes podem criar tópicos   & Sim & Sim & \textbf{Não}& Sim   & Sim    \\
	-inscrever-se em um fórum  & Sim & Sim & \textbf{Não}& Sim   & \textbf{Não}    \\
	-assinar tópicos& Sim & \textbf{Não} & \textbf{Não}& Sim   & \textbf{Não}     \\
	-moderar fórum& Sim & \textbf{Não}  & Sim& Sim   & \textbf{Não}     \\
	Pesquisar fóruns& Sim & Sim & Sim& Sim   & Sim {\tiny (busca geral)} \\
	 & & & & & \\
	\bottomrule
	\end{tabular}
	\end{tiny}
	\caption{Tabela de comparação categorias: Conteúdo e Atribuição}
	\label{tab:conteudo-atribuicao}
\end{table}

\begin{table}[H]
	\centering
	\begin{tiny}
	\begin{tabular}{C{4cm}C{3.2cm}C{3.2cm}C{3.2cm}C{3.2cm}C{3.2cm}}
	\toprule
	\textbf{Funcionalidades\textbackslash Sistema} & \textbf{BlackBoard 9.1} & \textbf{Moodle} & \textbf{TelEduc} & \textbf{Sakai} & \textbf{Noosfero}\\
	\midrule
	\multicolumn{6}{c}{\textbf{Curso}}\\
	\midrule
	Criar cursos  & Sim & Sim & Sim  & Sim   & Sim    \\
	Criar novos papéis& Sim & Sim & \textbf{Não} & \textbf{Não}    & Sim \\
	Ferramentas de grupos & Sim & Sim & Sim& Sim   & Sim \\
	Relatórios do curso   & Sim & Sim & Sim& Sim   & \textbf{Não}  \\
	Sistema de alerta & Sim & \textbf{Não}  & \textbf{Não} & \textbf{Não}    & \textbf{Não}  \\
	Personalização ou configuração (nome, duração, inscrição, idioma) & Sim & Sim & Sim& Sim   & Sim    \\
	Formato de curso(fórum, formato dos tópicos, semanal)& \textbf{Não}  & Sim & \textbf{Não} &  & \textbf{Não}     \\
	Gerenciar campo menu  & Sim & Sim {\tiny O menu é personalizado de acordo com conteúdo o curso} & \textbf{Não} & Sim   & Sim    \\
	Customizar estilo de curso & Sim {\tiny (menu, tema)} & Sim {\tiny(tema) }& \textbf{Não} &  & Sim\\
	Estrutura de pastas   & Sim & Sim & \textbf{Não} & Sim   & Sim    \\
	Gerenciar ferramentas & Sim & Sim & Sim& Sim   & Sim    \\
	Backup do curso / exportação & Sim & Sim & \textbf{Não} & Sim   & Sim\\
	Importar curso& Sim & Sim & \textbf{Não} & \textbf{Não}    & \textbf{Não}  \\
	Restaurar curso & Sim & Sim & \textbf{Não} & \textbf{Não}    & \textbf{Não}     \\
	Importar conteúdo/características de outros cursos   & Sim & Sim & Sim & Sim   & Sim    \\
	Gerenciador de arquivos    & Sim & Sim & \textbf{Não} & Sim   & Sim \\
	\bottomrule
	\end{tabular}
	\end{tiny}
	\caption{Tabela de comparação categoria Curso}
	\label{tab:curso}
\end{table}

\begin{table}[H]
	\centering
	\begin{tiny}
	\begin{tabular}{C{4cm}C{3.2cm}C{3.2cm}C{3.2cm}C{3.2cm}C{3.2cm}}
	\toprule
	\textbf{Funcionalidades\textbackslash Sistema} & \textbf{BlackBoard 9.1} & \textbf{Moodle} & \textbf{TelEduc} & \textbf{Sakai} & \textbf{Noosfero}\\
	\midrule
	\multicolumn{6}{c}{\textbf{Ferramentas}}\\
	\midrule
	Glossário & Sim {\tiny (Editável pelo instrutor)} & Sim {\tiny (Os participantes também podem adicionar entradas) + bloco para glossário} & \textbf{Não}  & Sim & \textbf{Não} \\
	RSS Feeds & \textbf{Não}  & Sim {\tiny (Banco de dados, Fórum, Glossário)} & \textbf{Não}  & Sim & \textbf{Não}  \\
	Blogs & Sim & Sim & Sim & Sim & Sim \\
	Lista de contatos & Sim & Sim & Sim & \textbf{Não}  & Sim {\tiny (lista de participantes)}\\
	Calendário (Curso) & Sim & Sim & Sim & Sim & {\tiny Apenas de eventos} \\
	Permissões de acesso & Sim {\tiny (programada, dependendo de grau, dependendo do nível total do curso, depende do acesso a conteúdo definido, definir suas próprias regras)} & Sim \tiny{(programada, dependendo de grau, dependendo do nível total do curso, depende do acesso a conteúdo definidos / atividades)} & Sim & Sim & Sim \\
	Verificação de plágio & Sim {\tiny (integrado)} & \textbf{Não}  {\tiny (Não integrado, mas existem APIs)} & \textbf{Não}  & \textbf{Não}  & \textbf{Não}  \\
	Busca de arquivos (curso) & Sim {\tiny (adicionando um bloco correspondente)} & \textbf{Não}  & Sim & Sim & Sim \\
	Jornal & Sim & \textbf{Não}  & \textbf{Não}  &  & \textbf{Não}  \\
	Atividade concluída & Sim & Sim & Sim & Sim & Sim {\tiny (\textbf{Não}  totalmente implementado)} \\
	Ferramentas de voz & \textbf{Não}  (é necessário a ferramenta Wimba instalada) & \textbf{Não}  & \textbf{Não}  & Sim & \textbf{Não}  \\
	Permite o uso de módulos e plugins & Sim & Sim & \textbf{Não}  & Sim {\tiny (permite a seleção de ferramentas} & Sim \\ 
	\bottomrule
	\end{tabular}
	\end{tiny}
	\caption{Tabela de comparação categoria Ferramentas}
	\label{tab:ferramentas}
\end{table}

\begin{table}[H]
	\centering
	\begin{tiny}
	\begin{tabular}{C{4cm}C{3.2cm}C{3.2cm}C{3.2cm}C{3.2cm}C{3.2cm}}
	\toprule
	\textbf{Funcionalidades\textbackslash Sistema} & \textbf{BlackBoard 9.1} & \textbf{Moodle} & \textbf{TelEduc} & \textbf{Sakai} & \textbf{Noosfero}\\
	\midrule
	\multicolumn{6}{c}{\textbf{Teste/Quiz}}\\
	\midrule
	Banco de questões & Sim & Sim & Sim & Sim & \textbf{Não}  \\
	Exportar resultados & Sim & Sim & Sim & \textbf{Não}  & \textbf{Não}  \\
	Exportar respostas & Sim & \textbf{Não}  & \textbf{Não}  & Sim & \textbf{Não}  \\
	\textit{\textbf{Tipo de questões}} &  &  &  &  &  \\
	resposta com arquivo & Sim & Sim & \textbf{Não}  & Sim & Sim \\
	resposta por escala & Sim & Sim {\tiny (múltipla escolha)} & \textbf{Não}  & \textbf{Não}  & \textbf{Não}  \\
	preencher espaço vazio & Sim & Sim {\tiny (difícil de lidar)} & \textbf{Não}  & Sim & \textbf{Não}  \\
	múltipla escolha & Sim & Sim & Sim & Sim & \textbf{Não}  \\
	resposta múltipla & Sim & Sim & Sim & Sim & Sim {\tiny(permite o envio de várias versões de um arquivo)} \\
	Verdadeiro/Falso & Sim & Sim & Sim & Sim & \textbf{Não}  \\
	Calculos & Sim & Sim & \textbf{Não}  & Sim & \textbf{Não}  \\
	Resposta dissertativa & Sim & Sim & Sim & Sim & Sim {\tiny (com a creiação de um artigo)} \\
	Calculo numérico & Sim & Sim & \textbf{Não}  & Sim & \textbf{Não}  \\
	\bottomrule
	\end{tabular}
	\end{tiny}
	\caption{Tabela de comparação categoria Teste/Quiz}
	\label{tab:teste}
\end{table}

\begin{table}[H]
	\centering
	\begin{tiny}
	\begin{tabular}{C{4cm}C{3.2cm}C{3.2cm}C{3.2cm}C{3.2cm}C{3.2cm}}
	\toprule
	\textbf{Funcionalidades\textbackslash Sistema} & \textbf{BlackBoard 9.1} & \textbf{Moodle} & \textbf{TelEduc} & \textbf{Sakai} & \textbf{Noosfero}\\
	\midrule
	\multicolumn{6}{c}{\textbf{Permissões e papéis}}\\
	\midrule
	Papéis pré-definidos & Sim {\tiny (instrutor, assistente de ensino, construtor curso, nivelador, estudante, convidado;)} & Sim {\tiny (Administrador do site, gerente, criador do curso, professor editor, professor, estudante, convidado, usuário;)} & Sim {\tiny (coordenadores, formadores, alunos, visitantes, colaboradores;)} & Sim {\tiny (instrutores, assistente de professor, estudante;)} & Sim {\tiny (Administrador do sistema, moderador, membro, administrador de perfil;)} \\
	Editar papéis existentes & Sim & Sim & \textbf{Não}  & \textbf{Não}  & Sim {\tiny(apenas com permissão)} \\
	Criar novos papéis & Sim & Sim & \textbf{Não}  & \textbf{Não}  & Sim \\
	\midrule
	\multicolumn{6}{c}{\textbf{Página principal}}\\
	\midrule
	RSS Feeds & \textbf{Não}  & Sim & \textbf{Não}  & Sim & \textbf{Não}  \\
	Avisos/ Novos fóruns & Sim & Sim & Sim & Sim & \textbf{Não}  \\
	Lista de afazeres/ Próximos eventos & Sim & Sim & Sim {\tiny(Agenda)} & Sim & Sim {\tiny(Calendário)} \\
	Mensagens & \textbf{Não}  & Sim & Sim & Sim & Sim \\
	Boletim/ Resultados de questionários & Sim & Sim & Sim & Sim & \textbf{Não}  \\
	Usuários online & Sim & Sim & Sim & Sim {\tiny(chat)} & \textbf{Não}  \\
	Meu calendário & Sim & Sim & Sim & Sim & Sim \\
	Baixar o conteúdo completo do curso & \textbf{Não}  & \textbf{Não}  & Sim {\tiny(todos os arquivos)} & Sim & \textbf{Não}  \\
	Informação pessoal/ Perfil de usuários & Sim & Sim & Sim & Sim & Sim \\
	Cancelar a inscrição de um curso & Sim & Sim & \textbf{Não}  & Sim & Sim \\
	Software livre - GPL license & \textbf{Não}  & Sim & Sim & Sim & Sim \\
	Extensão a plataformas móveis & Sim & \textbf{Não}  & \textbf{Não}  & \textbf{Não}  & \textbf{Não}  \\ 
	\bottomrule
	\end{tabular}
	\end{tiny}
	\caption{Tabela de comparação categorias: Permissões e papéis, página principal}
	\label{tab:permissoes-principal}
\end{table}

\end{landscape}

Os AVA mencionados possuem papéis de usuários pré-estabelecidos (Tabela \ref{tab:permissoes-principal}), apesar do Noosfero possuir estes papéis preestabelecidos possui a vantagem de criação de papéis customizados, onde o administrador define quais as permissões para cada um deles.

O Noosfero carece de uma melhor estruturação dos conteúdos para o gerenciamento de um curso. Conforme evidenciado na Tabela \ref{tab:curso}, a plataforma não permite tratar os conteúdos na forma de cursos e tampouco a personalização de seus formatos (formato dos tópicos, duração). Além disso a ferramenta de calendário disponível mostra apenas as datas dos eventos criados, não disponibilizando aos alunos todas as atividades e duração do curso de forma limpa e organizada.

Quando analisados os itens da categoria de página principal (Tabela \ref{tab:permissoes-principal}) é verificado que o Noosfero necessita de um template próprio para cursos que reúne as informações principais dos alunos. Algumas dessas informações necessitam apenas de uma melhor disposição pela interface, como calendário e mensagens. Já outras exigem implementação como um sistema de avisos, usuários online e boletim.

Neste capitulo foi apresentado o conceito de ambientes virtuais de aprendizagem, bem como as  suas funcionalidades e uma comparação com o Noosfero. Este estudo é importante, uma vez que considerando esta análise, algumas funcionalidades serão implementadas no desenvolvimento prático deste trabalho, como proposto na seção \ref{desen-noosferAVA}. No entanto, é necessário explorar o Noosfero (Capítulo \ref{evol-rede-social}) assim como os processos e desenvolvimento da comunidade. Essa compreensão é importante, uma vez que o Comunidade.UnB utiliza essa plataforma que será o foco principal desse trabalho.

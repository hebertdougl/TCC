\chapter{AVAs e Plataformas de redes sociais}
\label{avas-redes-sociais}
% 
\section{Ambiente virtual de aprendizagem}
\label{ava}
% 
No atual cenário em que vivemos a tecnologia da informação está cada vez mais presente em nossa rotina, seja com uso de celulares, tablets, televisores e etc. e utilizá-la no ambiente educacional pode ser uma grande vantagem para integrar à sociedade do conhecimento. Neste cenário ambientes virtuais de aprendizagem tem sido cada vez mais usados na mais diversas insituições (acadêmicas, empresariais e tecnológicas) com o objetivo de apoiar o ensino à distância.

Os ambientes virtuais de aprendizagem (AVAs)\footnote{\textit{Virtual Learning Environment} (VLEs)} são sistemas que tem a funcionalidade de \textit{software} para a comunicação mediada por computador e métodos de entrega de material de cursos online. Muito desses ambientes tem a possibilidade de simular a sala de aula física em um meio virtual contudo existem ambientes que além de realizar este papel fazem uso da tecnologia para propiciar aos envolvidos novas ferramentas que potencializem a comunicação e facilitem a aprendizagem \cite{schlemmer2005ambiente}.

% Esses últimos procuram suportar uma grande e variada gama de estilos de aprendizagem e objetivos, encorajando a colaboração, a aprendizagem baseada em pesquisa, além de promover compartilhamento e reuso dos recursos.

Os AVAs são utilizadas por várias instituições de ensino do mundo e o mercado apresenta uma grande variedade de \textit{softwares} sejam eles proprietários ou livres, mas apesar dessa diversidade \citeonline{behar2009avaliaccao} destaca que muitas características e ferramentas são comuns entre eles:
\begin{itemize}
\item permitem acesso restrito a usuários previamente cadastrados;
\item disponibilizam espaço para a publicação de material do professor;
\item disponibilizam espaço destinado ao envio/armazenamento de tarefas realizadas pelos alunos;
\item conjunto de ferramentas de comunicação síncrona e assíncrona\footnote{Como \textit{chat} e fórum de discussões};
\item correio eletrônico (\textit{e-mail});
\item mural de recados;
\end{itemize}
% Linkar parágrafos
Além dessas características \citeonline{aguado2013dimensoes} ressalta que os principais componentes encontrados consistem em:
\begin{itemize}
\item disponibilização de conteúdo de ensino;
\item mapeamento do conteúdo do curso (quebra do conteúdo em sessões que podem ser acessadas e cumpridas);
\item disponibilização de atividades de interação e avaliação;
\item acompanhamento do desempenho do aluno;
\item suporte online;
\item links de internet para conteúdos externos;
\item identificações e permissões de acesso.
\end{itemize}

Apesar dos autores supracitados relatarem as principais características e componentes dos AVAs, constata-se que nem todos oferecem as mesmas funcionalidades como pode ser verificado ainda neste trabalho na seção \ref{comparacao-ava}. O escopo das opções disponíveis nos sistemas nem sempre são as mesmas, evidenciando a razão do alto número de sistemas no mercado, pois os produtos possuem pontos fortes e pontos fracos, e por isso, nem sempre as funcionalidades atendem aos diversos tipos de instituições e seus respectivos objetivos \cite{aguado2013dimensoes}.

Segundo \citeonline{chin2003virtual}, as funcionalidades oferecidas pelo AVA facilitam e fortalecem o aprendizado de conceitos dados presencialmente. Mesmo que os AVAs apresentem características principais do ensino tradicional em sala de aula, \citeonline{chin2003virtual} propõe que os AVAs não devem ser vistos como uma substituição à aula presencial, uma vez que sempre deverá existir a interação humana física no processo educacional.

Dentre os vários sistemas dispooníveis no mercado alguns necessitam instalação no computador cliente, porém em sua maioria deles é acessível pela internet onde a instalação é realizada apenas no servidor. Das ferramentas encontradas atualmente destaca-se as tradicionalmente utilizadas como o Moodle ~\footnote{\textit{Modular Object-Oriented Dynamic Learning Environment}. Disponível em: \url{https://moodle.org/}} e o BlackBoard~\footnote{Disponível em: \url{http://www.blackboard.com/}} que são melhores apresentadas na seção \ref{comparacao-ava}.
%
%A expressão Ambiente Virtual de Aprendizagem, de acordo com \cite{almeidaEAD}, “relaciona-se à sistemas computacionais, destinados ao suporte de atividades mediadas pelas tecnologias de informação e comunicação”. Permitem integrar múltiplas mídias e recursos, apresentam informações de maneira organizada, proporcionam interações entre pessoas e objetos de conhecimento, visando atingir determinados objetivos.
%
% Dentro deste contexto, ao se tratar a qualidade dessas ferramentas é necessário considerar que o item mais importante a ser avaliado é o critério didático-pedagógico, afinal, este é um produto da área da educação que deve auxiliar os alunos e envolvidos a aquisição de algum conhecimento, porém não se pode deixar de lado outros aspectos importantes de usabilidade técnica que podem inviabilizar o uso da ferramenta ou mesmo, torna-lo difícil a ponto da aprendizagem deixar de ser prazerosa e assim a tecnologia não estará cumprindo seu principal papel que é ser um grande facilitador do processo de aprendizagem.
%
\section{Redes sociais}
\label{rede-social}

Para entender o conceito de redes sociais é importante destacar que são apenas uma camada das mídias sociais, já que é comum vermos a utilização deste termo para todos os tipos de mídias sociais mediadas por um computador. As tecnologias de mídias sociais \footnote{Termo em inglês, \textit{Social Media Technology} (SMT)} podem ser definidas de forma abrangente como a totalidade de produtos e serviços digitais disponibilizados online, como aplicações \textit{web} e \textit{mobile} que permitem aos indivíduos criar e compartilhar conteúdos onde sua principal fonte são os usuários, através da comunicação em várias vias \cite{davis2012social}.

Há várias definições entre os conceitos de \textit{rede}, desta maneira deve-se levar em consideração nesta monografia as definições proposta por \cite{emirbayer1994network}, onde rede é um conjunto de relações ou ligações entre um conjunto de atores, no caso atores são os elos às pessoas que se comunicam em uma dada rede. Além disso para este trabalho considerou-se que a rede “é uma forma de organização caracterizada fundamentalmente pela sua horizontalidade, isto é, pelo modo de inter-relacionar os elementos sem hierarquia” \cite[p. 73]{costa2004redes}. E segundo \citeonline{tomael2005redes} a rede, que é uma estrutura não-linear, descentralizada, flexível, dinâmica, sem limites definidos e autoorganizável, estabelece-se por relações horizontais de cooperação .

Segundo \cite{marteleto2001analise}, as redes sociais derivam dos conceitos de rede e representam um conjunto de participantes autônomos, que buscam unir suas idéias e recursos em torno de valores e interesses compartilhados. A autora ressalta a idéia de compartilhamento de valores e interesses que, para promover o fortalecimento da rede, dependem do compartilhamento da informação e do conhecimento.

Fundamentado nestes principais conceitos o foco deste trabalho é nos sites de redes sociais, os autores \citeonline{ellison2007social} definem sites de redes sociais como serviços \textit{web} que permitem a seus usuários criarem perfis, e por meio deles conectar com outros usuários, propiciando a busca e o cruzamento de informções dentro da sua lista de conexões.

Para este trabalho leva-se em consideração as redes sociais que tem como objetivo de promover a interação em torno das colaborações, isto é, apoiada no entendimento de redes de colaboração. Isto se aplica ao contexto de uma rede social de nicho de uma universidade, como explicitado na seção \ref{comunidade-unb} onde descreve-se a proposta de disponibilização de uma rede social para a Universidade de Brasília (UnB). A exemplo, nesta rede as pessoas entram na rede para acompanharem uma disciplina, um projeto ou um determinado grupo de trabalho dentro da universidade, de maneira que os alunos também podem explorar a rede para encontrar comunidades e conteúdos de seu interesse.

\citeonline{bucher2013rede} menciona que esta prática adotada por redes de colaboração não é possível não é possível em redes monopolistas e centralizadoras, porque em geral seus conteúdos estão dipostos de forma fragmentada ou são controlados. Isto demonstra que o uso de uma rede de colaboração dentro de universidades torna-se a escolha mais plausível para a difusão do conhecimento.

Apoiado nas vantagens proporcionadas pelas redes sociais de colaboração e sua implantação na UnB, busca-se neste trabalho associá-la aos conceitos dos ambientes virtuais de aprendizagem tratados na seção \ref{ava}. A proposta visa a criação de funcionalidades que viabilizem o uso de uma plataforma de redes sociais como um ambiente de apoio aos ambientes virtuais de aprendizagem.

\section{Comparação}
\label{comparacao-ava}

Como o objetivo do trabalho é a evolução de plataforma de redes sociais para um ambiente virtual de aprendizagem, a pesquisa realizada fundamenta-se no levantamento das principais funcionalidades que os ambientes virtuas de aprendizagem atuais utilizam para verificar quais a plataforma Noosfero ainda não possui. Desse modo surgiu a necessidade do levantamento das principais ferramentas AVAs utilizadas pelas universidades e instituições de ensino atualmente, para proporcionar o mapeamento destas funcionalidades.

Para a escolha das ferramentas estabeleceu-se apenas o critério de ser as que são atualmente utilizadas dentro e fora do Brasil, para tanto foram realizadas pesquisas  pela \textit{internet} e indicações de alunos da própria UnB que estudam em outras universidades no exterior. Deste modo foram definidas três principais ferramentas mundialmente utilizadas: o \textbf{BlackBoard}, \textbf{Sakai} e \textbf{Moodle}.

O BlackBoard~\footnote{Disponível em: \url{http://www.blackboard.com/}}, é um dos maiores ambientes virtuais de aprendizagem que são utilizados atualmente, onde um total de 72\% das TOP 200 melhores universidades do mundo à utilizam. Esta plataforma é de propriedade da Microsoft e para sua utilização é necessário a aquisição de licenças por parte da instituição que a pretende implementar. Nesse sentido a implantação dessa plataforma tem total suporte da empresa que a mantém onde a impossibilidade de modificação de sua estrutura interna pelos seus utilziadores, os ajustes e melhorias das funcionalidades e usabilidade são realizados através do mapeamento dos comentários e sugestões dos usuários.

O Moodle \footnote{Disponível em: \textit{ \url{https://moodle.org/}}}, \textit{Modular Object-Oriented Dynamic Learning Environment}, é um software livre criado em 2001 pelo educador cientista computacional Martin Dougiamas. É desenvolvido na linguagem de programação PHP e desta maneira pode ser instalado em qualquer sistema operacional que tenha suporte a linguagem. Para seu uso podem ser utilizadas várias bases de dados como MySQL, PostgreSQL, Oracle ou qualquer outro que suporte o padrão ODBC \footnote{Padrão para acesso a sistemas gerenciadores de bancos de dados}.

O projeto Sakai\footnote{Disponível em: https://sakaiproject.org/ } ou a comunidade Sakai desenvolve e distribui o software livre Sakai como ambiente virtual de aprendizagem para colaboração e ensino para educadores, por educadores. A plataforma foi desenvolvido em linguagem Java e pode ser executado em várias plataformas diferentes como Linux, Unix, Windows e MAC, suporta banco de dados MySQL e Oracle. As instituições que utilizam esta plataforma é bastante abrangente e pode ser obtida no site da comunidade \footnote{Disponível em: https://www.sakaiproject.org/community}.
% Solar
% Amadeus
% Colocar repositórios das comunidades?

E a última ferramenta escolhida, mas não menos importante, foi o \textbf{TelEduc}\footnote{Disponível em: \url{http://www.teleduc.org.br/}} é um ambiente virtual de aprendizagem que teve o início de seu desenvolvimento em 1997, a partir de uma dissertação de mestrado de Alessandra de Dutra e Cerceau do Instituto de Computação da Universidade Estadual de Campinas\footnote{IC/UNICAMP. Disponível em: \url{http://www.ic.unicamp.br/}}, sendo objeto de pesquisa e desenvolvimento de um projeto coordenado pela Profa Bra Heloísa Vieira da Rocha\footnote{\url{http://lattes.cnpq.br/6985892121344767}} até 2012. Desde de então, o projeto vem sendo evoluído afim de implementar ajustes e novas funcionalidades segundo \cite{rocha2002ambiente}, visando as necessidades relatadas por seus usuário. O projeto é mantido pelo Núcleo de Informática Aplicada à Educação \footnote{NIED/UNICAMP. Disponível em:\url{http://www.nied.unicamp.br/}} da UNICAMP.

O TelEduc é um software livre desenvolvido na linguagem de programção PHP \footnote{Disponível em: \url{http://php.net/}} e JavaScript com banco de dados o MySQL \footnote{Disponível em: \url{https://www.mysql.com/}} para ambientes Linux. Em seu lançamento tornou-se um dos softwares mais utilizados para apoiar a educação a distância nas mais diversas áreas, sendo que seus principais usuários são as universidades públicas e privadas, que utilizam para atividades educacionais, disponibilizando materiais, dando suporte a comunicação e interação entre os participantes.

A partir das ferramentas escolhidas criou-se uma tabela comparativa entre elas e a plataforma Noosfero, desse modo fica mais evidente de quais funcionalidades a plataforma Noosfero carece. Lembrando que não é objetivo deste trabalho esclarecer ou afirmar qual é a melhor ferramenta, mas verificar todas a principais funcionalidades de um AVA que são utilizados e aceitos pelas universidades afim de compará-las com o Noosfero.

Baseado na literatura apresentada na seção \ref{ava}, as funcionalidades das ferramentas escolhidas foram obtidas através de seu uso em ambientes onlines de demonstração, contando ainda, com o auxílio de mecanismos de ajuda fornecidos pelos mesmos. Assim sendo uma série de funcionalidades foram consideradas e para uma melhor apresentação desses dados foram divididas em oito categorias:
\begin{itemize}
\item \textbf{Conteúdo:} funcionalidades relacionadas à criação e manutenção de conteúdos publicados pelos usuários.
\item \textbf{Atribuição:} permitem que os usários carreguem arquivos do seu computador local para a plataforma.
\item \textbf{Ferramentas:} mecanismos de apoio ao gerenciamento do conteúdo e ambiente;
\item \textbf{Teste/Quiz:} permitem a realização de avaliações e acompanhamento de resultados dos alunos;
\item \textbf{Comunicação:} são ferramentas que possibilitam a comunicação síncrona e assíncrona entre os participantes de um curso;
\item \textbf{Curso:} são as funcionalidades fundamentais para organização dos cursos;
\item \textbf{Permissão e papéis:} funcionalidades de administração utilizadas para gerenciar o ambiente, definindo papéis e permissões ao ambiente;
\item \textbf{Página principal:} são as principais funcionalidades evidenciadas na página inicial de cada usuário.
\end{itemize}

Das categorias da tabela \ref{tab:conteudo-atribuicao} podemos destacar que o Noosfero carece do SCORM \footnote{Um modelo de referência seja, conjunto unificado de especificações para a disponibilização de conteúdos e serviços de e-learning \cite{de2006objetos}.} e IMS-Content-Package \footnote{Padrão que permite exportar o conteúdo de um sistema de gerenciamento de conteúdo de aprendizagem ou repositório digital.} que são funcionalidades importantes para os AVAs, pois permitem a importação de conteúdos de outras ferramentas proporcionando flexibilidade para o usuário.

Realizando uma análise da tabela \ref{tab:comunicacao} verifica-se que o Noosfero possui funcionalidades básicas de comunicação, não possibilitando apenas a comunicação síncrona (Bate-papo). O Noosfero possui fórums, que normalmente são utilizados para a divulgação de notícias e dúvidas, mas não permite a assinatura de tópicos para o recebimento de notificações.

Como o Noosfero não é um ambiente virtual de aprendizagem é possível verificar na tabela \ref{tab:teste} que ele não possibilita a criação de questões para obter a resposta de usuários. Da mesma forma que a plataforma não conta com um sistema de notas, permitindo ao professor atribuir as notas e acompanhar a situação de cada aluno. O Noosfero possui um \textit{plugin} denominado \textit{Work Assignment} que permite aos usuários enviarem arquivos para o servidor mas não possibilita que o moderador estabeleça uma pontuação para os mesmos.

\begin{landscape}
% conteudo atribuição
\begin{table}[H]
\begin{center}
\begin{tabular}{|@{}p{5.5cm}|p{3.5cm}|p{3.5cm}|p{3.5cm}|p{3.5cm}|p{3.5cm}@{}|}
\hline
\textbf{Funcionalidades\textbackslash Sistema} & \textbf{BlackBoard 9.1} & \textbf{Moodle} & \textbf{TelEduc} & \textbf{Sakai} & \textbf{Noosfero}\\ \hline
\textbf{Conteúdo} &  &  &  &  &  \\
Diretórios/Pastas & Sim & Sim & Sim & Sim & Sim \\
Criar/Extrair arquivos de pastas & Sim & Sim & Sim & Sim & Sim \\
Editor HTML & Sim & Sim & Sim & Sim & Sim \\
Filtro de vários idiomas & Não & Sim & Não & Não & Sim \\
Audio & Sim {\tiny (incorporado no plugin do QuickTime)} & Sim & Não & Sim &  \\
Vídeo & Sim & Sim & Não & Sim & Sim \\
SCORM & Sim & Sim & Não & Sim & Não \\
IMS-Content-Package & Sim & Sim &  &  & Não \\
\textbf{Atribuição} &  &  &  &  &  \\
Upload de um arquivo & Sim & Sim & Sim & Sim & Sim \\
Upload de vários arquivos & Sim & Sim & Não & Sim & Sim \\
Texto online & Sim(quiz) & Sim & Sim & Sim & Sim \\ \hline
\end{tabular}
\caption{Tabela de comparação categorias: Conteúdo e Atribuição}
\label{tab:conteudo-atribuicao}
\end{center}
\end{table}

% Curso e comunicação
\begin{table}[H]
\begin{tabular}{|@{}p{5.5cm}|p{3.5cm}|p{3.5cm}|p{3.5cm}|p{3.5cm}|p{3.5cm}@{}|}
\hline
\textbf{Funcionalidades\textbackslash Sistema} & \textbf{BlackBoard 9.1} & \textbf{Moodle} & \textbf{TelEduc} & \textbf{Sakai} & \textbf{Noosfero}\\ \hline
\textbf{Comunicação} & & & & & \\
Enviar e-mail & Sim & Sim & Sim& Sim   & Sim    \\
Enviar mensagens& Sim & Sim & Sim& Sim   & Sim {\tiny(mural)}\\
Bate papo& Sim & Sim & Sim& Sim   & Não    \\
Fórum de discussão& Sim & Sim & Sim& Sim   & Sim    \\
Alterar tipo de fórum & Sim & Sim & Sim& Sim   & Sim \\
Fórum:   & Sim & Sim & Sim& Sim   & Sim    \\
-postagem anônima & Sim & Não & Não& Não   & Não    \\
-anexar um arquivo& Sim & Sim & Não& Sim   & Não    \\
-os participantes podem criar tópicos   & Sim & Sim & Não& Sim   & Sim    \\
-inscrever-se em um fórum  & Sim & Sim & Não& Sim   & Não    \\
-assinar tópicos& Sim & Não & Não& Sim   & Não    \\
-moderar fórum& Sim & Não & Sim& Sim   & Não    \\
Pesquisar fóruns& Sim & Sim & Sim& Sim   & Sim {\tiny (busca geral)} \\
 & & & & & \\ \hline
\end{tabular}
\caption{Tabela de comparação categoria Comunicação}
\label{tab:comunicacao}
\end{table}

\begin{table}[H]
\begin{tabular}{|@{}p{5.5cm}|p{3.5cm}|p{3.5cm}|p{3.5cm}|p{3.5cm}|p{3.5cm}@{}|}
\hline
\textbf{Funcionalidades\textbackslash Sistema} & \textbf{BlackBoard 9.1} & \textbf{Moodle} & \textbf{TelEduc} & \textbf{Sakai} & \textbf{Noosfero}\\ \hline
\textbf{Curso}&&   &&  &   \\
Criar cursos  & Sim & Sim & Sim  & Sim   & Sim    \\
Criar novos papéis& Sim & Sim & Não& Não   & Sim \\
Ferramentas de grupos & Sim & Sim & Sim& Sim   & Sim \\
Relatórios do curso   & Sim & Sim & Sim& Sim   & Não \\
Sistema de alerta & Sim & Não & Não& Não   & Não \\
Personalização ou configuração (nome, duração, inscrição, idioma) & Sim & Sim & Sim& Sim   & Sim    \\
Formato de curso(fórum, formato dos tópicos, semanal)& Não & Sim & Não&  & Não    \\
Gerenciar campo menu  & Sim & Sim {\tiny O menu é personalizado de acordo com conteúdo o curso} & Não& Sim   & Sim    \\
Customizar estilo de curso & Sim {\tiny (menu, tema)} & Sim {\tiny(tema) }& Não&  & Sim\\
Estrutura de pastas   & Sim & Sim & Não& Sim   & Sim    \\
Gerenciar ferramentas & Sim & Sim & Sim& Sim   & Sim    \\
Backup do curso / exportação & Sim & Sim & Não& Sim   & Sim\\
Importar curso& Sim & Sim & Não& Não   & Não \\
Restaurar curso & Sim & Sim & Não& Não   & Não    \\
Importar conteúdo/características de outros cursos   & Sim & Sim & Sim & Sim   & Sim    \\
Gerenciador de arquivos    & Sim & Sim & Não& Sim   & Sim \\ \hline
\end{tabular}
\caption{Tabela de comparação categoria Curso}
\label{tab:curso}
\end{table}

% Ferramentas teste/quiz
\begin{table}[H]
\begin{tabular}{|@{}p{5.5cm}|p{3.5cm}|p{3.5cm}|p{3.5cm}|p{3.5cm}|p{3.5cm}@{}|}
\hline
\textbf{Funcionalidades\textbackslash Sistema} & \textbf{BlackBoard 9.1} & \textbf{Moodle} & \textbf{TelEduc} & \textbf{Sakai} & \textbf{Noosfero}\\ \hline
\textbf{Ferramentas} &  &  &  &  &  \\
Glossário & Sim(Editável pelo instrutor) & Sim {\tiny (Os participantes também podem adicionar entradas) + bloco para glossário} & Não & Sim &  \\
RSS Feeds & Não & Sim {\tiny (Banco de dados, Fórum, Glossário)} & Não & Sim & Não \\
Blogs & Sim & Sim & Sim & Sim & Sim \\
Lista de contatos & Sim & Sim & Sim & Não & Sim {\tiny (lista de participantes)}\\
Calendário (Curso) & Sim & Sim & Sim & Sim & {\tiny Apenas de eventos} \\
Permissões de acesso & Sim {\tiny (programada, dependendo de grau, dependendo do nível total do curso, depende do acesso a conteúdo definido, definir suas próprias regras)} & Sim \tiny{(programada, dependendo de grau, dependendo do nível total do curso, depende do acesso a conteúdo definidos / atividades)} & Sim & Sim & Sim \\
Verificação de plágio & Sim {\tiny (integrado)} & Não {\tiny (Não integrado, mas existem APIs)} & Não & Não & Não \\
Busca de arquivos (curso) & Sim {\tiny (adicionando um bloco correspondente)} & Não & Sim & Sim & Sim \\
Jornal & Sim & Não & Não &  & Não \\
Atividade concluída & Sim & Sim & Sim & Sim & Sim {\tiny (não totalmente implementado)} \\
Ferramentas de voz & Não (é necessário a ferramenta Wimba instalada) & Não & Não & Sim & Não \\
Permite o uso de módulos e plugins & Sim & Sim & Não & Sim {\tiny (permite a seleção de ferramentas} & Sim \\ \hline
\end{tabular}
\caption{Tabela de comparação categoria Ferramentas}
\label{tab:ferramentas}
\end{table}

\begin{table}[H]
\begin{tabular}{|@{}p{5.5cm}|p{3.5cm}|p{3.5cm}|p{3.5cm}|p{3.5cm}|p{3.5cm}@{}|}
\hline
\textbf{Funcionalidades\textbackslash Sistema} & \textbf{BlackBoard 9.1} & \textbf{Moodle} & \textbf{TelEduc} & \textbf{Sakai} & \textbf{Noosfero}\\ \hline
\textbf{Teste/Quiz} &  &  &  &  &  \\
Banco de questões & Sim & Sim & Sim & Sim & Não \\
Exportar resultados & Sim & Sim & Sim & Não & Não \\
Exportar respostas & Sim & Não & Não & Sim & Não \\
\textit{\textbf{Tipo de questões}} &  &  &  &  &  \\
resposta com arquivo & Sim & Sim & Não & Sim & Sim \\
resposta por escala & Sim & Sim {\tiny (múltipla escolha)} & Não & Não & Não \\
preencher espaço vazio & Sim & Sim {\tiny (difícil de lidar)} & Não & Sim & Não \\
múltipla escolha & Sim & Sim & Sim & Sim & Não \\
resposta múltipla & Sim & Sim & Sim & Sim & Sim {\tiny(permite o envio de várias versões de um arquivo)} \\
Verdadeiro/Falso & Sim & Sim & Sim & Sim & Não \\
Calculos & Sim & Sim & Não & Sim & Não \\
Resposta dissertativa & Sim & Sim & Sim & Sim & Sim {\tiny (com a creiação de um artigo)} \\
Calculo numérico & Sim & Sim & Não & Sim & Não \\ \hline
\end{tabular}
\caption{Tabela de comparação categoria Teste/Quiz}
\label{tab:teste}
\end{table}

\begin{table}[H]
\begin{tabular}{|@{}p{5.5cm}|p{3.5cm}|p{3.5cm}|p{3.5cm}|p{3.5cm}|p{3.5cm}@{}|}
\hline
\textbf{Funcionalidades\textbackslash Sistema} & \textbf{BlackBoard 9.1} & \textbf{Moodle} & \textbf{TelEduc} & \textbf{Sakai} & \textbf{Noosfero}\\ \hline
\textbf{Permissões e papéis} &  &  &  &  &  \\
Papéis pré-definidos & Sim & Sim & Sim & Sim & Sim \\
Editar papéis existentes & Sim & Sim & Não & Não & Sim {\tiny(apenas com permissão)} \\
Criar novos papéis & Sim & Sim & Não & Não & Sim \\ \hline
\textbf{Página principal} &  &  &  &  &  \\
RSS Feeds & Não & Sim & Não & Sim & Não \\
Avisos/ Novos fóruns & Sim & Sim & Sim & Sim & Não \\
Lista de afazeres/ Próximos eventos & Sim & Sim & Sim {\tiny(Agenda)} & Sim & Sim {\tiny(Calendário)} \\
Mensagens & Não & Sim & Sim & Sim & Sim \\
Boletim/ Resultados de questionários & Sim & Sim & Sim & Sim & Não \\
Usuários online & Sim & Sim & Sim & Sim {\tiny(chat)} & Não \\
Meu calendário & Sim & Sim & Sim & Sim & Sim \\
Baixar o conteúdo completo do curso & Não & Não & Sim {\tiny(todos os arquivos)} & Sim & Não \\
Informação pessoal/ Perfil de usuários & Sim & Sim & Sim & Sim & Sim \\
Cancelar a inscrição de um curso & Sim & Sim & Não &  &  \\
Software livre - GPL license & Não & Sim & Sim & Sim & Sim \\
Extensão a plataformas móveis & Sim & Não & Não & Não & Não \\ \hline
\end{tabular}
\caption{Tabela de comparação categorias: Permissões e papéis, página principal}
\label{tab:permissoes-principal}
\end{table}

\end{landscape}

Os AVAs mencionados possuem papéis de usuários pré-estabelecidos (Tabela \ref{tab:permissoes-principal}): BlackBoard: instrutor, assistente de ensino, construtor curso, nivelador, estudante, convidado; Moodle: Administrador do site, gerente, criador do curso, professor editor, professor, estudante, convidado, usuário; TelEduc: Coordenadores, formadores, alunos, visitantes, colaboradores; Sakai: instrutores, assistente de professor, estudante; Noosfero: Administrador do sistema, moderador, membro, administrador de perfil. Apesar do Noosfero possuir estes papéis pré estabelecidos possui a vantagem de criação de criação de papéis customizados, onde o administrador define quais as permissões para cada um deles. Esta funcionalidade foi desenvolvida recentemente e se encontra em aprovação do \textit{merge request}.

O Noosfero carece de uma melhor estruturação dos conteúdos para o gerenciamento de um curso, como evidenciado na tabela \ref{tab:curso}, a plataforma não permite tratar os conteúdos na forma de cursos e tampouco a personalização de seus formatos (formato dos tópicos, duração). Além disso a ferramenta de calendário disponível mostra apenas as datas dos eventos criados, não disponibilizando aos alunos todas as atividades e duração do curso de forma limpa e organizada.

Quando analisado os itens da categoria de página principal (Tabela \ref{tab:permissoes-principal}) é verificado que o Noosfero necessita de um template próprio para cursos que reúne as informações principais ao alunos. Algumas dessas informações necessitam apenas de uma melhor disposição pela interface como o calendário e mensagens, e outras como avisos, usuários online, boletim etc. requerem a implementação.

Neste capitulo vimos o conceito de ambientes virtuais de aprendizagem, bem como suas funcionalidades além de uma comparação com o Noosfero. Este estudo é importante, uma vez que considerando a análise dessas funcionalidades algumas serão implementadas no desenvolvimento prático deste trabalho, como proposto na seção \ref{desen-noosferAVA}. No entanto é necessário explorar o Noosfero (Capítulo \ref{evol-rede-social}) assim como os processos e desenvolvimento da comunidade, essa compreensão é importante, uma vez que o Comunidade.UnB utiliza esta plataforma que será o foco principal desse trabalho.
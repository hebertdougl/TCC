\chapter{Ambiente Virtual de Aprendizagem e Plataformas de Redes Sociais}
\label{avas-redes-sociais}

Neste capítulo serão apresentadas os conceitos relacionados aos ambientes virtuais de aprendizagem e redes sociais. Além disso há uma comparação dos principais AVA levantados nessa pesquisa com a plataforma Noosfero.

\section{Ambiente virtual de aprendizagem}
\label{ava}

No atual cenário, em que vivemos a tecnologia da informação está cada vez mais presente em nossa rotina, seja com o uso de celulares, tablets, televisores entre outros e utilizá-la no ambiente educacional pode ser uma grande vantagem de integração à sociedade do conhecimento. Nesse cenário ambientes virtuais de aprendizagem tem sido cada vez mais usados nas mais diversas instituições (acadêmicas, empresariais e tecnológicas) com o objetivo de apoiar o ensino à distância.

Os ambientes virtuais de aprendizagem \footnote{Em inglês \textit{Virtual Learning Environment} (VLE)} são sistemas que tem a funcionalidade de \textit{software} para a comunicação entre computador e métodos de entrega de material de cursos online. Muitos desses ambientes permitem simular a sala de aula física em um meio virtual contudo existem ambientes que além de realizar este papel fazem uso da tecnologia para propiciar aos envolvidos novas ferramentas que potencializem a comunicação e facilitem a aprendizagem \cite{schlemmer2005ambiente}.

% Esses últimos procuram suportar uma grande e variada gama de estilos de aprendizagem e objetivos, encorajando a colaboração, a aprendizagem baseada em pesquisa, além de promover compartilhamento e reuso dos recursos.

Os AVA são utilizadas por várias instituições de ensino do mundo. O mercado apresenta uma grande variedade de \textit{softwares} proprietários e livres. Entretanto, apesar dessa diversidade, \citeonline{behar2009avaliaccao} destaca que muitas características e ferramentas são comuns entre eles:
\begin{itemize}
\item permitem acesso restrito a usuários previamente cadastrados;
\item disponibilizam espaço para a publicação de material do professor;
\item disponibilizam espaço destinado ao envio/armazenamento de tarefas realizadas pelos alunos;
\item disponibilizam um conjunto de ferramentas de comunicação síncrona e assíncrona\footnote{Como \textit{chat} e fórum de discussões};
\item disponibilizam correio eletrônico (\textit{e-mail}) e mural de recados.
\end{itemize}

% Linkar parágrafos
Além dessas características, \citeonline{aguado2013dimensoes} ressalta que os principais componentes encontrados consistem de:
\begin{itemize}
\item disponibilização de conteúdo de ensino;
\item mapeamento do conteúdo do curso (quebra do conteúdo em sessões que podem ser acessadas e cumpridas);
\item disponibilização de atividades de interação e avaliação;
\item acompanhamento do desempenho do aluno;
\item suporte online;
\item links de internet para conteúdos externos;
\item identificações e permissões de acesso.
\end{itemize}

Apesar dos autores supracitados relatarem as principais características e componentes dos AVA, constata-se que nem todos oferecem as mesmas funcionalidades como pode ser verificado na seção \ref{comparacao-ava} desse trabalho. O escopo das opções disponíveis nos sistemas nem sempre é o mesmo, evidenciando a razão do alto número de sistemas no mercado, pois os produtos possuem pontos fortes e fracos, e por isso, nem sempre as funcionalidades atendem aos diversos tipos de instituições e seus respectivos objetivos \cite{aguado2013dimensoes}.

Segundo \citeonline{chin2003virtual}, as funcionalidades oferecidas pelo AVA facilitam e fortalecem o aprendizado de conceitos dados presencialmente. Mesmo que os AVA apresentem características principais do ensino tradicional em sala de aula, \citeonline{chin2003virtual} propõe que os AVA não devem ser vistos como substitutos à aula presencial, uma vez que sempre deverá existir a interação humana física no processo educacional.

Dentre os vários sistemas disponíveis no mercado alguns necessitam de instalação no computador cliente, porém, a maioria é acessível pela internet, onde a instalação é realizada apenas no servidor. Das ferramentas encontradas atualmente destacam-se as tradicionalmente utilizadas como o Moodle ~\footnote{\textit{Modular Object-Oriented Dynamic Learning Environment}. Disponível em: \url{https://moodle.org/}} e o BlackBoard~\footnote{Disponível em: \url{http://www.blackboard.com/}} detalhadas na seção \ref{comparacao-ava}.

\section{Redes sociais}
\label{rede-social}

Para entender o conceito de redes sociais é importante destacar que são apenas uma camada das mídias sociais. É comum vermos a utilização deste termo para todos os tipos de mídias sociais mediadas por um computador. As tecnologias de mídias sociais \footnote{Termo em inglês, \textit{Social Media Technology} (SMT)} podem ser definidas de forma abrangente como a totalidade de produtos e serviços digitais disponibilizados online, como aplicações \textit{web} e \textit{mobile} que permitem aos indivíduos criar e compartilhar conteúdos onde sua principal fonte são os usuários, através da comunicação em várias vias \cite{davis2012social}.

As definições de rede utilizadas neste trabalho estão relacionadas com três propostas. A primeira é de acordo com \citeonline{emirbayer1994network} que definem rede como um conjunto de relações ou ligações entre um conjunto de atores, no caso atores são os elos às pessoas que se comunicam em uma dada rede. No mesmo sentido, \citeonline[p. 73]{costa2004redes} a define com uma forma de organização caracterizada fundamentalmente pela sua horizontalidade, isto é, pelo modo de inter-relacionar os elementos sem hierarquia. E por fim, mas não menos importante, faz-se uso da definição proposta por \citeonline{tomael2005redes} no qual rede é uma estrutura não-linear, descentralizada, flexível, dinâmica, sem limites definidos e autoorganizáveis. Destacando que uma rede estabelece-se por relações horizontais de cooperação.

Segundo \cite{marteleto2001analise}, as redes sociais derivam dos conceitos de rede e representam um conjunto de participantes autônomos, que buscam unir suas idéias e recursos em torno de valores e interesses compartilhados. A autora ressalta a ideia de compartilhamento de valores e interesses que, para promover o fortalecimento da rede, dependem do compartilhamento da informação e do conhecimento.

Nesse contexto, os autores \citeonline{ellison2007social} definem sites de redes sociais como serviços \textit{web} que permitem a seus usuários criarem perfis e, por meio deles, conectarem com outros usuários, propiciando a busca e o cruzamento de informações dentro da sua lista de conexões.

Fundamentado nestes principais conceitos, este trabalho está focado em sites de redes sociais, que possuem o objetivo de promover a interação em torno das colaborações, isto é, apoiada no entendimento de redes de colaboração. Isto se aplica ao contexto de uma rede social de nicho de uma universidade, como explicitado na seção \ref{comunidade-unb} onde é descrita a proposta de disponibilização de uma rede social para a Universidade de Brasília (UnB). A rede da UnB permite que as pessoas entrem na rede para acompanharem uma disciplina, um projeto ou um determinado grupo de trabalho. Os alunos também podem explorar a rede para encontrar comunidades e conteúdos de seus interesses.

\citeonline{bucher2013rede} menciona que esta prática adotada por redes de colaboração não é possível em redes monopolistas e centralizadoras, porque em geral seus conteúdos estão dipostos de forma fragmentada ou são controlados. Isto demonstra que o uso de uma rede de colaboração dentro de universidades é uma escolha mais plausível para a difusão do conhecimento.

Apoiado nas vantagens proporcionadas pelas redes sociais de colaboração e sua implantação na UnB, busca-se neste trabalho associá-las aos conceitos dos ambientes virtuais de aprendizagem tratados na Seção \ref{ava}. A proposta visa a criação de funcionalidades que viabilizem o uso de uma plataforma de redes sociais como um ambiente de apoio aos ambientes virtuais de aprendizagem.

Neste capitulo foi apresentado os principais conceitos de ambientes virtuais de aprendizagem e redes sociais, no entanto é necessário explorar o Noosfero (Capítulo \ref{evol-rede-social}) assim como os seus processos e desenvolvimento dentro da comunidade de \textit{software} livre. Esse conhecimento é essencial, uma vez que utilizaremos o Comunidade.UnB, que utiliza essa plataforma sendo o foco principal desse trabalho.

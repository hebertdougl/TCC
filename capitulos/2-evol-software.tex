\chapter{Manutenção e evolução de software}
\label{cap-evol-software}

Neste capítulo será abordado a manutenção e evolução de software provendo uma comparação entre os métodos tradicionais e empíricos empregados na engenharia de software, evidenciando algumas práticas adotadas nas comunidades de \textit{software} livre, que serão utilizadas neste trabalho.

\section{Manutenção e evolução de software nos métodos tradicionais}
\label{sect-manut-evol-soft}

A \citeonline{iso12207} 12207 define o ciclo de vida de um software por meio de agrupamento de processos que são divididos em quatro classes:fundamentais, apoio, organizacionais e adaptação. O processo de manutenção de software está classificado nessa ISO como um dos processos fundamentais,ou seja necessários para que um o software seja implementado.

% Falar mais sobre processos fundamentais.

O Instituo de Engenheiros Eletricistas e Eletrônicos \citeonline{ieee1219}, define manutenção de software como a modificação de um produto de software após sua entrega com vistas a corrigir falhas, melhorar o desempenho ou outros atributos, e adaptar o produto a um ambiente modificado de acordo com as necessidades do cliente.

As normas citadas tratam a manutenção de software como se o mesmo fosse um produto genérico. Para \citeonline[p.~380]{pfleeger2004engenharia} a manutenção varia de acordo com o propósito de criação do software. Ainda, segundo a Autora, a probabilidade de o sistema ser modificado varia de acordo com o vínculo de dependência dos requisitos com o sistema do mundo real, pois, quanto mais dependentes forem os requisitos do mundo real maior será a probabilidade do sistema ser modificado.

Para determinar o tipo de manutenção a ser realizado, \citeonline{pfleeger2004engenharia} separa sistemas em três grupos. O primeiro grupo de sistemas é o que possui os requisitos especificados e formalmente definidos, ou seja, os resultados esperados são bem conhecidos e a solução é estática e não se adapta facilmente a uma mudança no problema que o gerou. O segundo grupo consiste dos sistemas, cujas soluções são aproximadas do mundo real, tem-se a solução completa apenas na teoria pois solucionar o problema em sua totalidade é impraticável ou impossível. O jogo de xadrez é um exemplo desse grupo, uma vez que é impossível calcular todas os movimentos possíveis e suas consequências em tempo hábil para a próxima jogada. O terceiro grupo incorpora a mudança na natureza do mundo real em si,  modificando-se a medida em que o mundo muda. Tem como base um modelo do processo abstrato envolvido. Exemplo: sistema que prevê a estabilidade econômica de um país e utiliza como base o modelo econômico em uso.

% Verificar referencia IEE

Analisando a literatura percebe-se que a manutenção de software está dividida em três grandes categorias, listadas a seguir, que podem ser visualizados tanto na norma da \citeonline{ieee1219} e em livros, desde \citeonline{lientz1980software} à \citeonline{pfleeger2004engenharia}, as categorias são as seguintes:
\begin{itemize}
\item \textbf{Manutenção corretiva:} realizada para controlar funções cotidianos do sistema, corrigindo falhas e defeitos de funcionalidades do programa.

\item \textbf{Manutenção adaptativa:} visa a implementação de modificações secundárias que têm como origem uma modificação em outra parte do sistema ou do meio externo.

\item \textbf{Manutenção perfectiva ou evolutiva:} objetivam realizar acréscimo de novas funcionalidades ou mudanças no \textit{software} independente da existência de defeitos.
\end{itemize}

\citeonline{pressman2011engenharia} e \citeonline{criscuolo2008qualidade} acrescentam a manutenção preventiva. Essa está ligada a uma reengenharia, aplicando conceitos de engenharia de software para tornar os softwares mais fáceis de serem corrigidos, adaptados e melhorados ao longo dos anos. Para \citeonline{criscuolo2008qualidade} a manutenção preventiva permite que o software receba alterações com objetivo de minimizar os riscos de defeitos e reduzir a complexidade de sua estrutura, garantindo um certa segurança à manutenção.

%Manutenção preventiva (reengenharia): modifica o software a fim de torná-los mais fáceis de serem corrigidos, adaptados e melhorados. 

Para o \citeonline{ieee1219} além das categorias citadas, há a manutenção ``emergencial'', definida pelo IEEE como uma manutenção corretiva que não foi programada ou planejada, mas que deve ser executada para manter o sistema em funcionamento.

As atividades do processo de desenvolvimento de software ainda são dependentes da participação e criatividade humana, e o crescimento de métodos ágeis e software livre mostram que o código é um dos principais artefatos. Diante disso torna-se necessário na Engenharia de \textit{Software} utilizar métodos empíricos que apoiam o desenvolvimento de \textit{software} de acordo com sua natureza abstrata.

\section{Manutenção e evolução de software em métodos empíricos de desenvolvimento}

A ciência empírica diz respeito à aquisição de conhecimentos através de métodos empíricos. No entanto, o que constitui o conhecimento,assim sendo, os métodos para adquiri-lo baseia-se em pressupostos básicos sobre ontologia (ou seja, o que acreditamos existir) e epistemologia (ou seja, como as crenças são adquiridas e que justifica-los). Dessa maneira a pesquisa empírica procura explorar, descrever, prever e explicar fenômenos naturais, sociais, cognitivas ou usando evidências baseadas em observação ou experiência. Trata-se de obtenção e interpretação de evidências, por exemplo: experimentação; observação sistemática; entrevistas ou inquéritos; ou pelo análise minuciosa de documentos ou artefatos \cite{sjoberg2007future}.

Neste capítulo serão abordadas algumas metodologias e práticas no processo de desenvolvimento de software que advém de métodos empíricos, dentre elas destaca-se os métodos ágeis e o processo de desenvolvimento de software livre.

\subsection{Software Livre}
\label{soft-livre}

De acordo com \citeonline{meirelles2013} software expressa uma solução abstrata dos problemas computacionais, neste contexto é o componente que contém o conhecimento relacionado aos problemas a que a computação se aplica, contendo diversos aspectos que ultrapassam questões técnicas, tais como:
\begin{itemize}
\item o processo de desenvolvimento do software;
\item os mecanismos econômicos que regem o desenvolvimento e o uso do software, a exemplo de gerenciais, competitivos, sociais, cognitivos etc.;
\item o relacionamento entre desenvolvedores, fornecedores e usuários do software;
\item os aspectos éticos e legais relacionados ao software.
\end{itemize}

O que define e diferencia o software livre de software proprietário vai do entendimento desses quatro pontos dentro do que é conhecido como \textit{ecossistema do software livre}~\cite{meirelles2013}.

Software livre é definido pela Free Software Foundation como “uma questão de liberdade dos usuários para executar, copiar, distribuir, estudar, mudar e melhorar o software. Levando isto em consideração significa que os usuários do software possuem quatro liberdades essenciais \cite{stallman2002free}:

\begin{itemize}
\item executar o programa, para qualquer propósito (liberdade no. 0);
\item estudar como o programa funciona, e adaptá-lo para as suas necessidades (liberdade no. 1). Aceso ao código-fonte é um pré-requisito para esta liberdade;
\item redistribuir cópias de para ajudar o próximo (liberdade no. 2);
\item aperfeiçoar o programa, e liberar os seus aperfeiçoamentos, de modo que toda a comunidade se beneficie (liberdade no. 3). Acesso ao código-fonte é um pré-requisito para esta liberdade.
\end{itemize}

Para que as liberdades um e três façam sentido, é necessário que o desenvolvedor ou pesquisador tenha acesso ao código-fonte do programa.  Portanto, acesso ao código-fonte é uma condição necessária ao software livre ~\cite{gnu2013}.

Um programa é considerado um software livre se os usuários possuem todas as liberdades citadas. Dessa forma, você deve ser livre para redistribuir cópias, com ou sem modificações, de graça ou cobrando uma taxa pela distribuição, para qualquer lugar. Ser livre para fazer essas coisas significa, entre outras coisas, que não há necessidade de pedir ou pagar por permissão para fazê-las ~\cite{anaPaula2012}.

O fato de que o código-fonte pode ser livremente compartilhado oferece vantagens ao software livre em comparação ao software proprietário. Umas delas é simplificação do desenvolvimento de aplicações personalizadas já que não precisam ser programadas a partir do zero, ou seja, podem se basear em soluções existentes.

A outra vantagem resultante do compartilhamento do código refere-se à possibilidade de melhoria na qualidade, em particular frente aos problemas inerentes à sua complexidade \cite{catedralBazzar}.

Isso se deve ao maior número de desenvolvedores e usuários envolvidos com o software, permitindo maiores situações de uso em necessidades variadas, o que propicia a identificação de um número maior de \emph{bugs} e mais sugestões de melhoria, promovendo refatorações que geralmente levam a melhoria do código.

O GNU teve a necessidade de utilizar termos de distribuição que impedissem a transformação de software livre em proprietário. O método usado foi chamado de \emph{Copyleft}. \emph{Copyleft} é um método geral para desenvolver um software livre e exige que todas as versões modificadas e estendidas do programa também sejam livres. Ele utiliza lei de direitos autorais, com finalidade diferente da usual: ao invés de um meio de privatizar o software, torna-se um meio de manter o software livre~\cite{stallman2009}.

\subsection{Relações entre métodos ágeis e processo de desenvolvimento de Software Livre}
\label{des-soft-livre}

Os métodos ágeis assim como processo de desenvolvimento de software livre, são métodos empíricos de software que provêm de um conjunto de metodologias baseada na prática de desenvolvimento de software. Em 2001 a aliança ágil definiu o Manifesto ágil, que definem preferências e príncipios que se espera de qualquer método de desenvolvimento dessa categoria. O manifesto é baseado em quatro valores \cite{beck2001agile}:
\begin{itemize}
\item Indivíduos e interações são mais importantes que processos e ferramentas.
\item Software em funcionamento é mais importante que documentação abrangente.
\item Colaboração com o cliente (usuários) é mais importante que negociação de contratos.
\item Responder às mudanças é mais importante que seguir um plano.
\end{itemize}

Na dissertação de \citeonline{corbucci2011metodos} há evidencias de que esses quatro valores demonstram a semelhança entre as práticas comuns utilizadas pelas comunidades de software livre e equipes ágeis. Além disso, o Autor, destaca que muitas práticas disseminadas pelas metodologias ágeis são usadas no dia a dia dos desenvolvedores e equipes das comunidades de software livre:
\begin{itemize}
\item código compartilhado (coletivo);
\item projeto simples;
\item repositório único de código;
\item integração contínua;
\item código e teste;
\item desenvolvimento dirigido por testes;
\item refatoração.
\end{itemize}

Entender esses aspectos no contexto de software livre torna-se relevante porque, ao contrário dos métodos tradicionais de desenvolvimento, o ágeis são adaptativos ao invés de prescritivos e são orientados às pessoas ao invés dos processos. Analisar o processo de desenvolvimento de software livre do ponto de vista da Engenharia de Software e as possíveis sinergias com os métodos ágeis podem contribuir para um melhor entendimento dessa disposição na criação e colaboração em torno de projetos de software livre \cite{meirelles2013}.

No processo de desenvolvimento de software livre, após a divulgação e lançamento da primeira versão do projeto, a solução está pronta para uso. Os usuários são desenvolvedores podem colaborar com o projeto afim de evoluí-lo para suprir necessidades pessoais ou de interesse da comunidade,são enviadas aos mantenedores do projeto como \textit{patches}\footnote{arquivos que contém as modificações no código}. Os mantenedores analisarão as propostas, e caso concordem com a mudança e implementação, irão aplicá-las ao repositório oficial do projeto. Portanto, mesmo que em projetos maiores outros aspectos sejam levados em consideração ou sigam processos mais burocráticos de colaboração, a essência da colaboração técnica está no envio e análise de trechos de código-fonte \cite{meirelles2013}.

As comunidades de software livre são ``livres'' para escolher e adotar as práticas que mais se adaptam ao seu modelo, dessa maneira é relatado o processo de desenvolvimento do Noosfero, um exemplo de desenvolvimento de software livre evidenciado na seção \ref{proc-desenvol-comunidade} do capítulo \ref{evol-rede-social}.

\subsubsection{Práticas baseadas em disciplinas da engenharia de software}

% Justificar que apesar dos modelos tradiionais dizerem uma coisa, na prática no contexto de software livre utilizamos isso

No contexto de gerência e configuração de software grande parte dos projetos de software livre atualmente utilizam algum sistema de controle de versão, visto como uma extensão natural do processo de desenvolvimento. Esse sistema, normalmente faz-se uso de um repositório central, onde são armazenadas todas as versões dos arquivos, e permite paralelizar o desenvolvimento de forma organizada, principalmente quando há um grande número de desenvolvedores \cite{reis2001caracterizaccao}. Permite ainda o \textit{commiter} oficial realizar inspeção e revisão de todo código que é integrado ao repositório principal do projeto de forma não intrusiva no desenvolvimento realizado pelos colaboradores. \citeonline{reis2001caracterizaccao} atestam que a revisão acaba acontecendo quase automaticamente, já que as pessoas são forçadas pela distância a enviar modificações a outras para a integração.

%    -Testes(GERAL)
O teste de software é um processo destinado a certificar-se de que o código desenvolvido é previsível e consistente,e que realiza apenas as ações para o qual foi projetado \cite{myers2011art}. Teste automatizado é um prática de criar \textit{scripts} ou programas de computador que exercitam o sistema em teste, capturam os efeitos colaterais e fazem verificações, de forma automática e dinâmica \cite{meszaros2007xunit}.

% teste automatizado
Segundo \citeonline{bernardo2011padroes} os testes automatizados agregam valor ao produto final porque afetam diretamente a qualidade dos sistemas de software, mesmo que os artefatos produzidos não sejam visíveis aos usuários.
% integração contínua

A integração contínua tem por objetivo integrar o sistema e todas suas dependências para assegurar que nenhuma modificação tenha danificado o sistema, sejam elas alterações no código-fonte, em configurações ou mesmo em dependências e outros fatores externos \cite{duvall2007continuous}. Desse modo, para certificar a integridade do código a integração contínua garante que todos os testes são executados a cada modificação da base de código fornecendo \textit{Feedback} imediato, assegurando que a integração tenha ocorrido de forma satisfatória.

\citeonline{corbucci2011metodos} afirma que a equipe utiliza os testes para descobrir eventuais defeitos o mais rapidamente possível, o que facilita e acelera a correção, além de reduzir a probabilidade de que pequenos problemas se transformarem em \textit{bugs} ou comportamentos inesperados no futuro.

% issue tracker
% Certamente o que se observa nas listas de discussão é que grande parte das mensagens é repleta de justificativas técnicas para defender um ponto de vista.

%    -Revisão

Baseado nas disciplinas de gerência e configuração e teste, fazendo uso de controle de versão, testes automatizados e integração contínua, é que evoluíremos a plataforma Noosfero com as funcionalidades selecionadas para melhor adaptar a plataforma com recursos de AVA. Para isso será abordado no próximo capítulo (\ref{avas-redes-sociais}) os recursos utilizados nos AVA comparando-os com o Noosfero.
\begin{resumo}[Abstract]
  \begin{otherlanguage*}{english}  

This study of Software Engineering aims to develop a platform of social
networking, with the scope to allow the addition of features of a Virtual
Learning Environment (VLE), proposing a hybridism between these environments.
In this context,  the free collaboration network Comunidade.UnB was chosen, an
environment for students and teachers to work in horizontal and collaborative
way, making use of Brazilian platform for free social networks, Noosfero. Thus,
the study included a survey of the main features of these learning environments
by comparing them with Noosfero, providing the implementation of a set of
features and improvements, in order to solve the needs raised. This study shows
the use of the proposed plugin architecture for the platform in question, and
understand how to perform the evolution of the platform in the empirical
development of software. Also explains the collaborative development process
in an open source community, sharing the knowledge gained with a team of
developers at UnB Gama.

  \vspace{\onelineskip}
 
  \noindent 
  \textbf{Key-words}: social networking. virtual learning environment. noosfero. agile methods. open-source software.
  \end{otherlanguage*}
\end{resumo}

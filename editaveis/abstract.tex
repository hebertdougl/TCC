\begin{resumo}[Abstract]
  \begin{otherlanguage*}{english}  

This Course Conclusion Paper of Software Engineering aims to evolve a social network platform to allow functions of a Virtual Learning Environment (VLE). For this, was chosen a open collaborative network Comunidade.Unb, a environment for students and teachers to work collaboratively, making use of a Brazilian platform  for open social networks, Noosfero. This way, was held a survey about the main features of these environments comparing them with Noosfero, and understand how to perform the evolution of the platform in the empirical software development. Therefore , is planned to contribute to the implementation of features that evolve the Comunidade.Unb to contain VLE resources, proposing a hybridism between these environments.

  \vspace{\onelineskip}
 
  \noindent 
  \textbf{Key-words}: social networking. virtual learning environment. noosfero. agile methods. open-source software.
  \end{otherlanguage*}
\end{resumo}

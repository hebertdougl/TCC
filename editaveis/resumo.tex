\begin{resumo}

Este trabalho de conclusão de curso de engenharia de software tem como objetivo evoluir uma plataforma de redes sociais, para permitir que ele contenha funcionalidades de um Ambiente Virtual de Aprendizagem (AVA), propondo um hibridismo entre esses ambientes. Neste contexto, escolheu-se utilizar a rede colaboração livre Comunidade.UnB, um ambiente para alunos e professores trabalharem de maneira horizontal e colaborativa, que faz uso da plataforma brasileira para redes sociais livre Noosfero. Dessa maneira, o estudo contemplou um levantamento sobre os principais recursos desses ambientes de aprendizagem comparando-os com o Noosfero, proporcionando a implementação de um conjunto de funcionalidades e melhorias, de maneira que atendesse as necessidades levantadas. Este estudo evidencia a utilização da arquitetura de \textit{plugins} proposta pela plataforma em questão, além de compreender como realizar a evolução da plataforma no desenvolvimento empírico de software. Além disso foi explanado o processo de desenvolvimento colaborativo em uma comunidade de software livre, compartihando o conhecimento adiquirido com uma equipe de desenvolvedores na UnB Gama.

 \vspace{\onelineskip}
    
 \noindent
 \textbf{Palavras-chaves}: redes sociais. ambiente virtual de aprendizagem. noosfero. métodos ágeis. software livre.
\end{resumo}
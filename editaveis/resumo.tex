\begin{resumo}

Este trabalho de conclusão de curso de engenharia de software tem como objetivo evoluir uma plataforma de redes sociais, para permitir que ele contenha funcionalidades de um Ambiente Virtual de Aprendizagem (AVA). Neste contexto, escolheu-se utilizar a rede colaboração livre Comunidade.UnB, um ambiente para alunos e professores trabalharem de maneira horizontal e colaborativa, que faz uso da plataforma brasileira para redes sociais livre Noosfero. Dessa maneira, realizou-se um levantamento sobre os principais recursos desses ambientes comparando-os com o Noosfero, além de compreender como realizar a evolução da plataforma no desenvolvimento empírico de software. Assim sendo, planeja-se contribuir com a implementação de funcionalidades, que evolua o Comunidade.Unb, para conter recursos de AVA, propondo um hibridismo entre esses ambientes.

 \vspace{\onelineskip}
    
 \noindent
 \textbf{Palavras-chaves}: redes sociais. ambiente virtual de aprendizagem. noosfero. métodos ágeis. software livre.
\end{resumo}